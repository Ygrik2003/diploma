\chapter{Результаты}
\begin{figure}[ht]
    \includegraphics{data/couette_react_error_best.png}
    \caption{Лучший результат в процессе кроссвалидации с функцией активации REAct}
    \label{fig:couette_react_best}
\end{figure}


В результате первого этапа обучения модель с предсказанным минимальным отклонением от
верного решения (рис. \ref{fig:couette_react_best}) имела следующие значения
гиперпараметров (табл. \ref{table:couette_react_best_params}).

\begin{table}[h!]
    \centering
    \caption{Значения гиперпараметров у модели с лучшим результатом}
    \begin{tabular}{ |c|c| } 
        \hline
        Функция активации & $\text{REAct}(0.8, -4, -3.2, 0.2)$ \\
        \hline
        Оптимизатор & Adagrad \\ 
        \hline
        Конфигурация сети & $64-16-32$ \\ 
        \hline
        Скорость обучения & $10^{-3}$ \\ 
        \hline
        Количество точек внутри области & $100$ \\ 
        \hline
    \end{tabular}
    \label{table:couette_react_best_params}
\end{table}

В результате анализа остальных моделей было подсчитано количество нулевых
решений, решений с максимальным отклонением до 20\%, а также больше 20\% и 50\%.
Помимо перечисленных также было несколько моделей, ушедших в процессе обучения в
NaN (рис. \ref{fig:couette_react_stat}). 

\begin{figure}[ht]
    \centering
    \begin{tikzpicture}
        \pgfplotstableread{
        Label   First
        NaN                     4
        Нулевое\ решение        30
        $>50\%$                 16
        $>20\%$                 94
        $<20\%$                 17
        }\datatable
        
        \begin{axis}[
            xbar stacked,
            xmin=0,
            ytick=data,
            yticklabels from table={\datatable}{Label}
            ]
            \addplot [fill=yellow] table [x=First, y expr=\coordindex] {\datatable};
        \end{axis}
    \end{tikzpicture}
    \caption{Распределение результатов кроссвалидации с функцией активации REAct}
    \label{fig:couette_react_stat}
\end{figure}

Также стоит отметить, что все нулевые решения соответствовали функции активации,
которая на всей области определения имеет отрицательные значения.

В качестве корректных параметров были отобраны те, которые соответствуют критерию
принадлежности к группе с уровнем менее 20\%. В результате, во второй этап были
включены следующие параметры:
\begin{enumerate}
    \item Конфигурации с резким переходом между слоями ($16-64-32$ и $64-16-32$) а также
    с линейным переходом ($16-32-64$ и $64-32-16$) не показали особых результатов по
    сравнению с остальными трехслойными сетями. Аналогично конфигурации
    $16-16$ и $32-32$ показали себя хуже по сравнению с $64-64$.
    \item По количеству точек внутри области значительно больше удачных
    результатов было при $100$ точек. Это поведение характеризуется 
    соотношением точек на границе и внутри области. Данное соотношение 
    показывает значимость граничных условий, что необходимо для избегания
    нулевого решения
    \item Скорость обучения влияет на то, сможет ли оптимизатор 
    выбраться из локального минимума. Лучший результат показало значение
    скорости обучения равное $10^{-3}$
    \item Среди оптимизаторов лучшие результаты показали \textbf{Adam}, \textbf{Adagrad} и
    \textbf{ASGD}, в свою очередь \textbf{Adamax} и \textbf{RSMProp} являются узкоспециализированными,
    что требует точной настройки параметров, что в данной работе не рассматривается.
\end{enumerate} 

Теперь получив более точное представление о наших параметрах можно приступить
ко второму этапу.

В представленных ниже графиках показано сравнение результатов работы нейронной сети
при различных гиперпараметрах. Такой подход позволяет детально проанализировать
влияние каждого параметра на общую эффективность модели и выявить, какие параметры
обеспечивают наилучшее качество обучения.
% \section{Зависимость от конфигурации нейронной сети}

Сперва стоит подметить, что в данном случае основными компонентами данной задачи
являются функции потерь на нижней и верхней границе в силу условий поставленной
задачи и только потом в учет идут уравнения Навье-Стокса. В большинстве случаев,
отдавая приоритет любой другой компоненте мы получим нулевое решение, что не
удовлетворяет поставленной задачи.

\begin{figure}[ht]
    \centering
    \begin{subfigure}[b]{0.4\textwidth}
        \begin{tikzpicture}[scale=0.85]
            \begin{axis}[
                ymode=log,
                legend style={font=\tiny},
                xmin=0,
                xtick distance=4000,
                axis lines=left,
                grid=both
            ]            
                \addplot+[mark=*, mark size=1pt, thick, red] table[x=step, y=value, col sep=comma]{data/couette_abu/loss/bc_bottom_neurons_(32, 64, 32).csv};
                \addlegendentry{(32, 64, 32)}
                \addplot+[mark=*, mark size=1pt, thick, green] table[x=step, y=value, col sep=comma]{data/couette_abu/loss/bc_bottom_neurons_(64, 32, 64).csv};
                \addlegendentry{(64, 32, 64)}
                \addplot+[mark=*, mark size=1pt, thick, blue] table[x=step, y=value, col sep=comma]{data/couette_abu/loss/bc_bottom_neurons_(64, 64).csv};
                \addlegendentry{(64, 64)}
                \addplot+[mark=*, mark size=1pt, thick, orange] table[x=step, y=value, col sep=comma]{data/couette_abu/loss/bc_bottom_neurons_(128, 128).csv};
                \addlegendentry{(128, 128)}
            \end{axis}
        \end{tikzpicture}
        \caption{Нижняя граница}
        \label{fig:bc_bottom_neurons}
    \end{subfigure}
    \hspace{0.5cm}
    \begin{subfigure}[b]{0.4\textwidth}
        \begin{tikzpicture}[scale=0.85]
            \begin{axis}[
                ymode=log,
                legend style={font=\tiny},
                xmin=0,
                xtick distance=4000,
                axis lines=left,
                grid=both
            ]
                \addplot+[mark=*, mark size=1pt, thick, red] table[x=step, y=value, col sep=comma]{data/couette_abu/loss/bc_top_neurons_(32, 64, 32).csv};
                \addlegendentry{(32, 64, 32)}
                \addplot+[mark=*, mark size=1pt, thick, green] table[x=step, y=value, col sep=comma]{data/couette_abu/loss/bc_top_neurons_(64, 32, 64).csv};
                \addlegendentry{(64, 32, 64)}
                \addplot+[mark=*, mark size=1pt, thick, blue] table[x=step, y=value, col sep=comma]{data/couette_abu/loss/bc_top_neurons_(64, 64).csv};
                \addlegendentry{(64, 64)}
                \addplot+[mark=*, mark size=1pt, thick, orange] table[x=step, y=value, col sep=comma]{data/couette_abu/loss/bc_top_neurons_(128, 128).csv};
                \addlegendentry{(128, 128)}
            \end{axis}
        \end{tikzpicture}
        \caption{Верхняя граница}
        \label{fig:bc_top_neurons}
    \end{subfigure}
    \begin{subfigure}[b]{0.4\textwidth}
        \begin{tikzpicture}[scale=0.85]
            \begin{axis}[
                ymode=log,
                legend style={font=\tiny},
                xmin=0,
                xtick distance=4000,
                axis lines=left,
                grid=both
            ]
                \addplot+[mark=*, mark size=1pt, thick, red] table[x=step, y=value, col sep=comma]{data/couette_abu/loss/bc_left_neurons_(32, 64, 32).csv};
                \addlegendentry{(32, 64, 32)}
                \addplot+[mark=*, mark size=1pt, thick, green] table[x=step, y=value, col sep=comma]{data/couette_abu/loss/bc_left_neurons_(64, 32, 64).csv};
                \addlegendentry{(64, 32, 64)}
                \addplot+[mark=*, mark size=1pt, thick, blue] table[x=step, y=value, col sep=comma]{data/couette_abu/loss/bc_left_neurons_(64, 64).csv};
                \addlegendentry{(64, 64)}
                \addplot+[mark=*, mark size=1pt, thick, orange] table[x=step, y=value, col sep=comma]{data/couette_abu/loss/bc_left_neurons_(128, 128).csv};
                \addlegendentry{(128, 128)}
            \end{axis}
        \end{tikzpicture}
        \caption{Левая граница}
        \label{fig:bc_left_neurons}
    \end{subfigure}
    \hspace{0.5cm}
    \begin{subfigure}[b]{0.4\textwidth}
        \begin{tikzpicture}[scale=0.85]
            \begin{axis}[
                ymode=log,
                legend style={font=\tiny},
                xmin=0,
                xtick distance=4000,
                axis lines=left,
                grid=both
            ]
                \addplot+[mark=*, mark size=1pt, thick, red] table[x=step, y=value, col sep=comma]{data/couette_abu/loss/bc_right_neurons_(32, 64, 32).csv};
                \addlegendentry{(32, 64, 32)}
                \addplot+[mark=*, mark size=1pt, thick, green] table[x=step, y=value, col sep=comma]{data/couette_abu/loss/bc_right_neurons_(64, 32, 64).csv};
                \addlegendentry{(64, 32, 64)}
                \addplot+[mark=*, mark size=1pt, thick, blue] table[x=step, y=value, col sep=comma]{data/couette_abu/loss/bc_right_neurons_(64, 64).csv};
                \addlegendentry{(64, 64)}
                \addplot+[mark=*, mark size=1pt, thick, orange] table[x=step, y=value, col sep=comma]{data/couette_abu/loss/bc_right_neurons_(128, 128).csv};
                \addlegendentry{(128, 128)}
            \end{axis}
        \end{tikzpicture}
        \caption{Правая граница}
        \label{fig:bc_right_neurons}
    \end{subfigure}
    \caption{Зависимость медианы функции потерь на каждой эпохе при разной конфигурации нейронной сети}
    \label{fig:bc_loss_neurons}
\end{figure}

Как можно заметить, на верхней границе (рис. \ref{fig:bc_top_neurons}), стабильнее
всего показывает себя результат при конфигурации нейронной сети $128-128$. В
случае трехслойных конфигураций можно подметить нестабильное поведение, что
может говорить о постоянном попадании в локальные минимумы. Такие конфигурации
нам не подходят в силу комплексности нашей функции потерь, где любая
нестабильность может привести к нулевому решению. Конфигурация $64-64$
на интервале до 4000 эпох не имеет перепадов, но ее результаты хуже,
чем у конфигурации $128-128$. Такая конфигурация может быть полезна, 
если нужно получить не очень точные результаты за короткий промежуток
времени.

%%%%%%%%%%%%%%%%%%%%%%%%%%%%%%%%%%%%%%%%%%%%%%%%%%%%%%%%%%%%%%%%%%%%%%%%%%%%%%

\begin{figure}[H]
    \centering
    \begin{subfigure}[b]{0.4\textwidth}
        \centering
        \begin{tikzpicture}[scale=0.85]
            \begin{axis}[
                ymode=log,
                ymin=1e-4, ymax=1e-2,
                legend style={font=\tiny},
                xmin=0,
                xtick distance=4000,
                axis lines=left,
                grid=both
            ]
                \addplot+[mark=*, mark size=1pt, thick, red] table[x=step, y=value, col sep=comma]{data/couette_abu/loss/pde_momentum_x_neurons_(32, 64, 32).csv};
                \addlegendentry{(32, 64, 32)}
                \addplot+[mark=*, mark size=1pt, thick, green] table[x=step, y=value, col sep=comma]{data/couette_abu/loss/pde_momentum_x_neurons_(64, 32, 64).csv};
                \addlegendentry{(64, 32, 64)}
                \addplot+[mark=*, mark size=1pt, thick, blue] table[x=step, y=value, col sep=comma]{data/couette_abu/loss/pde_momentum_x_neurons_(64, 64).csv};
                \addlegendentry{(64, 64)}
                \addplot+[mark=*, mark size=1pt, thick, orange] table[x=step, y=value, col sep=comma]{data/couette_abu/loss/pde_momentum_x_neurons_(128, 128).csv};
                \addlegendentry{(128, 128)}
            \end{axis}
        \end{tikzpicture}
        \caption{Уравнение для $u_x$}
        \label{fig:pde_ux_neurons}
    \end{subfigure}
    \hspace{0.5cm}
    \begin{subfigure}[b]{0.4\textwidth}
        \centering
        \begin{tikzpicture}[scale=0.85]
            \begin{axis}[
                ymode=log,
                ymin=1e-4, ymax=1e-2,
                legend style={font=\tiny},
                xmin=0,
                xtick distance=4000,
                axis lines=left,
                grid=both
            ]
                \addplot+[mark=*, mark size=1pt, thick, red] table[x=step, y=value, col sep=comma]{data/couette_abu/loss/pde_momentum_y_neurons_(32, 64, 32).csv};
                \addlegendentry{(32, 64, 32)}
                \addplot+[mark=*, mark size=1pt, thick, green] table[x=step, y=value, col sep=comma]{data/couette_abu/loss/pde_momentum_y_neurons_(64, 32, 64).csv};
                \addlegendentry{(64, 32, 64)}
                \addplot+[mark=*, mark size=1pt, thick, blue] table[x=step, y=value, col sep=comma]{data/couette_abu/loss/pde_momentum_y_neurons_(64, 64).csv};
                \addlegendentry{(64, 64)}
                \addplot+[mark=*, mark size=1pt, thick, orange] table[x=step, y=value, col sep=comma]{data/couette_abu/loss/pde_momentum_y_neurons_(128, 128).csv};
                \addlegendentry{(128, 128)}
            \end{axis}
        \end{tikzpicture}
        \caption{Уравнение для $u_y$}
        \label{fig:pde_uy_neurons}
    \end{subfigure}
    \begin{subfigure}[b]{0.4\textwidth}
        \centering
        \begin{tikzpicture}[scale=0.85]
            \begin{axis}[
                ymode=log,
                xmin=0, ymax=1e-2,
                legend style={font=\tiny},
                xtick distance=4000,
                axis lines=left,
                grid=both,
            ]
                \addplot+[mark=*, mark size=1pt, thick, red] table[x=step, y=value, col sep=comma]{data/couette_abu/loss/pde_continuity_neurons_(32, 64, 32).csv};
                \addlegendentry{(32, 64, 32)}
                \addplot+[mark=*, mark size=1pt, thick, green] table[x=step, y=value, col sep=comma]{data/couette_abu/loss/pde_continuity_neurons_(64, 32, 64).csv};
                \addlegendentry{(64, 32, 64)}
                \addplot+[mark=*, mark size=1pt, thick, blue] table[x=step, y=value, col sep=comma]{data/couette_abu/loss/pde_continuity_neurons_(64, 64).csv};
                \addlegendentry{(64, 64)}
                \addplot+[mark=*, mark size=1pt, thick, orange] table[x=step, y=value, col sep=comma]{data/couette_abu/loss/pde_continuity_neurons_(128, 128).csv};
                \addlegendentry{(128, 128)}
            \end{axis}
        \end{tikzpicture}
        \caption{Уравнение непрерывности}
        \label{fig:pde_continuity_neurons}
    \end{subfigure}
    \caption{Функция потерь для уравнений Навье-Стокса \eqref{eq:navier_stockes} при разной конфигурации нейронной сети}
    \label{fig:pde_loss_neurons}
\end{figure}

В добавок можно подметить небольшие флуктуации, возникающие в районе 5000 эпох. Это может свидетельствовать
об переобучении, которого также стоит избегать. В случае задач, где примерный результат известен, не составит
труда отличить верное решение от неверного. При таком исходе флуктуации могут помочь найти еще более
точное решение. Однако если нет возможности оценить правильность результата визуально --- стоит полагаться
только на стабильные участки функции потерь.

Рассмотрим нижнюю границу (рис. \ref{fig:bc_bottom_neurons}). На этой границе в поставленной задаче должно
выполняться равенство нулю всех компонент, что может выполняться в случае нулевого решения. Исключить 
нулевое решение можно оценив поведение уравнений Навье-Стокса для данной модели. В случае, если на верхней
границе функция потерь стремится к нулю и при этом функция потерь уравнения непрерывности (рис. \ref{fig:pde_continuity_neurons})
также стремится к нулю, можно сделать вывод о том, что решение не является нулевым.

Теперь когда мы знаем, что наше решение не нулевое --- осталось проанализировать внутренних переходов скорости,
за которые отвечают функции потери для компонент скорости в уравнениях Навье-Стокса (рис. \ref{fig:pde_ux_neurons}
и \ref{fig:pde_uy_neurons}). Для нашей задачи компонента $u_y$ малозначима в силу ее нулевого значения для всего
решения. Если рассматривать компоненту $v_x$ а также уравнение непрерывности, можно заметить разделение на два
кластера --- трехслойные и двухслойные модели. Пусть трехслойные и показывают лучшие результаты, чем двухслойные,
как мы выяснили ранее, такие модели могут чаще выдавать нулевое решение, что полностью соответствует малым значением
потерь для уравнений Навье-Стокса.


%%%%%%%%%%%%%%%%%%%%%%%%%%%%%%%%%%%%%%%%%%%%%%%%%%%%%%%%%%%%%%%%%%

\begin{figure}[H]
    \centering
    \begin{tikzpicture}
        \begin{axis}[
            width=0.6\textwidth,
            ymode=log,
            xmin=0,
            xtick distance=2000,
            axis lines=left,
            grid=both,
        ]
            \addplot+[mark=*, mark size=1pt, thick, red] table[x=step, y=value, col sep=comma]{data/couette_abu/loss/total_loss_neurons_(32, 64, 32).csv};
            \addlegendentry{(32, 64, 32)}
            \addplot+[mark=*, mark size=1pt, thick, green] table[x=step, y=value, col sep=comma]{data/couette_abu/loss/total_loss_neurons_(64, 32, 64).csv};
            \addlegendentry{(64, 32, 64)}
            \addplot+[mark=*, mark size=1pt, thick, blue] table[x=step, y=value, col sep=comma]{data/couette_abu/loss/total_loss_neurons_(64, 64).csv};
            \addlegendentry{(64, 64)}
            \addplot+[mark=*, mark size=1pt, thick, orange] table[x=step, y=value, col sep=comma]{data/couette_abu/loss/total_loss_neurons_(128, 128).csv};
            \addlegendentry{(128, 128)}
        \end{axis}
    \end{tikzpicture}
    \caption{Итоговая функция потерь при разной конфигурации нейронной сети}
    \label{fig:total_loss_neurons}
\end{figure}

Итого на суммарной функции потерь (рис. \ref{fig:total_loss_neurons}),
конфигурация $128-128$ имеет наименьшую функцию потерь на всех эпохах.
% \section{Зависимость от оптимизатора}

С точки зрения оптимизатора никаких ограничений на поведение
модели нет. От выбора оптимизатора зависит на сколько сложно
модели будет выбраться из локального минимума и приблизиться
к глобальному.

Рассмотрим верхнюю границу (рис \ref{fig:bc_top_optimizer}).
Исходя из графика, при использовании оптимизатора ASGD
функция потерь в среднем стремится к $~0.25$. При такой 
функции потерь максимальное отклонение от точного решения
может достигать $50\%$, если не учитывать полностью
нулевые решения. Сам по себе ASGD является мощным оптимизатором,
но требует тонкой настройки своих параметров, поэтому он
показывает худший результат в данной задаче.

\begin{figure}[H]
    \centering
    \begin{subfigure}[b]{0.4\textwidth}
        \begin{tikzpicture}[scale=0.85]
            \begin{axis}[
                ymode=log,
                legend style={font=\tiny},
                xmin=0,
                xtick distance=4000,
                axis lines=left,
                grid=both
            ]            
                \addplot+[mark=*, mark size=1pt, thick, red] table[x=step, y=value, col sep=comma]{data/couette_abu/loss/bc_bottom_optimizer_1.csv};
                \addlegendentry{Adam}
                \addplot+[mark=*, mark size=1pt, thick, green] table[x=step, y=value, col sep=comma]{data/couette_abu/loss/bc_bottom_optimizer_2.csv};
                \addlegendentry{Adagrad}
                \addplot+[mark=*, mark size=1pt, thick, blue] table[x=step, y=value, col sep=comma]{data/couette_abu/loss/bc_bottom_optimizer_4.csv};
                \addlegendentry{ASGD}
            \end{axis}
        \end{tikzpicture}
        \caption{Нижняя граница}
        \label{fig:bc_bottom_optimizer}
    \end{subfigure}
    \hspace{0.5cm}
    \begin{subfigure}[b]{0.4\textwidth}
        \begin{tikzpicture}[scale=0.85]
            \begin{axis}[
                ymode=log,
                legend style={font=\tiny},
                xmin=0,
                xtick distance=4000,
                axis lines=left,
                grid=both
            ]
                \addplot+[mark=*, mark size=1pt, thick, red] table[x=step, y=value, col sep=comma]{data/couette_abu/loss/bc_top_optimizer_1.csv};
                \addlegendentry{Adam}
                \addplot+[mark=*, mark size=1pt, thick, green] table[x=step, y=value, col sep=comma]{data/couette_abu/loss/bc_top_optimizer_2.csv};
                \addlegendentry{Adagrad}
                \addplot+[mark=*, mark size=1pt, thick, blue] table[x=step, y=value, col sep=comma]{data/couette_abu/loss/bc_top_optimizer_4.csv};
                \addlegendentry{ASGD}
            \end{axis}
        \end{tikzpicture}
        \caption{Верхняя граница}
        \label{fig:bc_top_optimizer}
    \end{subfigure}
    \begin{subfigure}[b]{0.4\textwidth}
        \begin{tikzpicture}[scale=0.85]
            \begin{axis}[
                ymode=log,
                legend style={font=\tiny},
                xmin=0,
                xtick distance=4000,
                axis lines=left,
                grid=both
            ]
                \addplot+[mark=*, mark size=1pt, thick, red] table[x=step, y=value, col sep=comma]{data/couette_abu/loss/bc_left_optimizer_1.csv};
                \addlegendentry{Adam}
                \addplot+[mark=*, mark size=1pt, thick, green] table[x=step, y=value, col sep=comma]{data/couette_abu/loss/bc_left_optimizer_2.csv};
                \addlegendentry{Adagrad}
                \addplot+[mark=*, mark size=1pt, thick, blue] table[x=step, y=value, col sep=comma]{data/couette_abu/loss/bc_left_optimizer_4.csv};
                \addlegendentry{ASGD}
            \end{axis}
        \end{tikzpicture}
        \caption{Левая граница}
        \label{fig:bc_left_optimizer}
    \end{subfigure}
    \hspace{0.5cm}
    \begin{subfigure}[b]{0.4\textwidth}
        \begin{tikzpicture}[scale=0.85]
            \begin{axis}[
                ymode=log,
                legend style={font=\tiny},
                xmin=0,
                xtick distance=4000,
                axis lines=left,
                grid=both
            ]
                \addplot+[mark=*, mark size=1pt, thick, red] table[x=step, y=value, col sep=comma]{data/couette_abu/loss/bc_right_optimizer_1.csv};
                \addlegendentry{Adam}
                \addplot+[mark=*, mark size=1pt, thick, green] table[x=step, y=value, col sep=comma]{data/couette_abu/loss/bc_right_optimizer_2.csv};
                \addlegendentry{Adagrad}
                \addplot+[mark=*, mark size=1pt, thick, blue] table[x=step, y=value, col sep=comma]{data/couette_abu/loss/bc_right_optimizer_4.csv};
                \addlegendentry{ASGD}
            \end{axis}
        \end{tikzpicture}
        \caption{Правая граница}
        \label{fig:bc_right_optimizer}
    \end{subfigure}
    \caption{Зависимость медианы функции потерь на каждой эпохе при разных оптимизаторах}
    \label{fig:bc_loss_optimizer}
\end{figure}

Похожая ситуация с Adagrad оптимизатором, без качественной
настройки параметров скорость обучения адаптивно уменьшается
и оптимизатор застывает в локальном минимуме. Вероятнее всего
при большем числе эпох Adagrad сможет догнать Adam, но нас
данный вариант не устраивает.

Оптимизатор Adam показывает лучший результат. Данный оптимизатор
включает в себя преймущества двух предыдущих и является универсальным,
поэтому не требует такой же точной настройки параметров.

Аналогичное поведение оптимизаторов можно заметить на графике для нижней
границы (рис. \ref{fig:bc_bottom_optimizer}). На графиках левой
(рис. \ref{fig:bc_left_optimizer}) и правой (рис. \ref{fig:bc_right_optimizer})
границах можно заметить смещение оптимизатора ASGD ближе к Adagrad, что может
свидетельствовать о преобладающем нулевом решении.

%%%%%%%%%%%%%%%%%%%%%%%%%%%%%%%%%%%%%%%%%%%%%%%%%%%%%%%%%%%%%%%%%%%%%%%%%%%%%%

\begin{figure}[H]
    \centering
    \begin{subfigure}[b]{0.4\textwidth}
        \centering
        \begin{tikzpicture}[scale=0.85]
            \begin{axis}[
                ymode=log,
                ymax=1e-2,
                legend style={font=\tiny},
                xmin=0,
                xtick distance=4000,
                axis lines=left,
                grid=both
            ]
                \addplot+[mark=*, mark size=1pt, thick, red] table[x=step, y=value, col sep=comma]{data/couette_abu/loss/pde_momentum_x_optimizer_1.csv};
                \addlegendentry{Adam}
                \addplot+[mark=*, mark size=1pt, thick, green] table[x=step, y=value, col sep=comma]{data/couette_abu/loss/pde_momentum_x_optimizer_2.csv};
                \addlegendentry{Adagrad}
                \addplot+[mark=*, mark size=1pt, thick, blue] table[x=step, y=value, col sep=comma]{data/couette_abu/loss/pde_momentum_x_optimizer_4.csv};
                \addlegendentry{ASGD}
            \end{axis}
        \end{tikzpicture}
        \caption{Уравнение для $u_x$}
        \label{fig:pde_ux_optimizer}
    \end{subfigure}
    \hspace{0.5cm}
    \begin{subfigure}[b]{0.4\textwidth}
        \centering
        \begin{tikzpicture}[scale=0.85]
            \begin{axis}[
                ymode=log,
                ymax=1e-2,
                legend style={font=\tiny},
                xmin=0,
                xtick distance=4000,
                axis lines=left,
                grid=both
            ]
                \addplot+[mark=*, mark size=1pt, thick, red] table[x=step, y=value, col sep=comma]{data/couette_abu/loss/pde_momentum_y_optimizer_1.csv};
                \addlegendentry{Adam}
                \addplot+[mark=*, mark size=1pt, thick, green] table[x=step, y=value, col sep=comma]{data/couette_abu/loss/pde_momentum_y_optimizer_2.csv};
                \addlegendentry{Adagrad}
                \addplot+[mark=*, mark size=1pt, thick, blue] table[x=step, y=value, col sep=comma]{data/couette_abu/loss/pde_momentum_y_optimizer_4.csv};
                \addlegendentry{ASGD}
            \end{axis}
        \end{tikzpicture}
        \caption{Уравнение для $u_y$}
        \label{fig:pde_uy_optimizer}
    \end{subfigure}
    \begin{subfigure}[b]{0.7\textwidth}
        \centering
        \begin{tikzpicture}[scale=0.85]
            \begin{axis}[
                ymode=log,
                ymax=1e-2,
                legend style={font=\tiny},
                xmin=0,
                xtick distance=4000,
                axis lines=left,
                grid=both
            ]
                \addplot+[mark=*, mark size=1pt, thick, red] table[x=step, y=value, col sep=comma]{data/couette_abu/loss/pde_continuity_optimizer_1.csv};
                \addlegendentry{Adam}
                \addplot+[mark=*, mark size=1pt, thick, green] table[x=step, y=value, col sep=comma]{data/couette_abu/loss/pde_continuity_optimizer_2.csv};
                \addlegendentry{Adagrad}
                \addplot+[mark=*, mark size=1pt, thick, blue] table[x=step, y=value, col sep=comma]{data/couette_abu/loss/pde_continuity_optimizer_4.csv};
                \addlegendentry{ASGD}
            \end{axis}
        \end{tikzpicture}
        \caption{Уравнение непрерывности}
        \label{fig:pde_continuity_optimizer}
    \end{subfigure}
    \caption{Функция потерь для уравнений Навье-Стокса \eqref{eq:navier_stockes} при разных оптимизаторах}
    \label{fig:pde_loss_optimizer}
\end{figure}

Для уравнений Навье-Стокса можно заметить рост функции потерь для оптимизатора
ASGD (рис. \ref{fig:pde_ux_optimizer} и \ref{fig:pde_continuity_optimizer}).
Таким образом происходит процесс поиска глобального минимума. Дело в том,
что функция потерь для верхней границы много больше, чем для уравнений Навье-Стокса
($10^{-0.8}$ против $10^{-2.9}$). Оптимизатор пытается выбраться из локального 
минимума, где решение стремится к нулевому в силу своей корректности с точки
зрения уравнений Навье-Стокса. Что касательно уравнения для скорости $u_y$
(рис. \ref{fig:pde_uy_optimizer}), график оптимизатора ASGD остается
практически неизменным, что опять же соответствует нулевому решению.
Остальные оптимизаторы имеют поведение схожее с поведением на границах.


%%%%%%%%%%%%%%%%%%%%%%%%%%%%%%%%%%%%%%%%%%%%%%%%%%%%%%%%%%%%%%%%%%

\begin{figure}[H]
    \centering
    \begin{tikzpicture}
        \begin{axis}[
            width=0.6\textwidth,
            ymode=log,
            xmin=0,
            xtick distance=2000,
            axis lines=left,
            grid=both,
        ]
            \addplot+[mark=*, mark size=1pt, thick, red] table[x=step, y=value, col sep=comma]{data/couette_abu/loss/total_loss_optimizer_1.csv};
            \addlegendentry{Adam}
            \addplot+[mark=*, mark size=1pt, thick, green] table[x=step, y=value, col sep=comma]{data/couette_abu/loss/total_loss_optimizer_2.csv};
            \addlegendentry{Adagrad}
            \addplot+[mark=*, mark size=1pt, thick, blue] table[x=step, y=value, col sep=comma]{data/couette_abu/loss/total_loss_optimizer_4.csv};
            \addlegendentry{ASGD}
        \end{axis}
    \end{tikzpicture}
    \caption{Итоговая функция потерь при разных оптимизаторах}
    \label{fig:total_loss_optimizer}
\end{figure}

Итого на суммарной функции потерь (рис. \ref{fig:total_loss_optimizer}),
оптимизатор Adam имеет наименьшую функцию потерь.
\section{Зависимость от фукции активации ABU}

В первую очередь функция активация играет основополагающую роль для
детерминирования поведения внутри домена. Это связано с тем, что
уравнения Навье-Стокса имеют сложную структуру. Если поставленная
задача имеет не нулевое решение, то нахождение верного решения внутри
домена зависит только от функции активации. Это означает, что нет смысла
ориентироваться на функцию потерь на границах.

Как ранее упоминалось, ABU является взвешенной суммой элементарных функций
активации. Рассмотрим влияние каждого слагаемого на функцию потерь.
\subsection{Квадратичная функция}
\begin{figure}[ht]
    \centering
    \begin{subfigure}[b]{0.4\textwidth}
        \begin{tikzpicture}[scale=0.85]
            \begin{axis}[
                ymode=log,
                legend style={font=\tiny},
                xmin=0,
                xtick distance=4000,
                axis lines=left,
                grid=both
            ]            
                \addplot+[mark=*, mark size=1pt, thick, red] table[x=step, y=value, col sep=comma]{data/couette_abu/loss/bc_bottom_scale_quadratic_0.0.csv};
                \addlegendentry{$0.0$}
                \addplot+[mark=*, mark size=1pt, thick, green] table[x=step, y=value, col sep=comma]{data/couette_abu/loss/bc_bottom_scale_quadratic_1.0.csv};
                \addlegendentry{$1.0$}
            \end{axis}
        \end{tikzpicture}
        \caption{Нижняя граница}
        \label{fig:bc_bottom_scale_quadratic}
    \end{subfigure}
    \hspace{0.5cm}
    \begin{subfigure}[b]{0.4\textwidth}
        \begin{tikzpicture}[scale=0.85]
            \begin{axis}[
                ymode=log,
                legend style={font=\tiny},
                xmin=0,
                xtick distance=4000,
                axis lines=left,
                grid=both
            ]
                \addplot+[mark=*, mark size=1pt, thick, red] table[x=step, y=value, col sep=comma]{data/couette_abu/loss/bc_top_scale_quadratic_0.0.csv};
                \addlegendentry{$0.0$}
                \addplot+[mark=*, mark size=1pt, thick, green] table[x=step, y=value, col sep=comma]{data/couette_abu/loss/bc_top_scale_quadratic_1.0.csv};
                \addlegendentry{$1.0$}
            \end{axis}
        \end{tikzpicture}
        \caption{Верхняя граница}
        \label{fig:bc_top_scale_quadratic}
    \end{subfigure}
    \begin{subfigure}[b]{0.4\textwidth}
        \begin{tikzpicture}[scale=0.85]
            \begin{axis}[
                ymode=log,
                legend style={font=\tiny},
                xmin=0,
                xtick distance=4000,
                axis lines=left,
                grid=both
            ]
                \addplot+[mark=*, mark size=1pt, thick, red] table[x=step, y=value, col sep=comma]{data/couette_abu/loss/bc_left_scale_quadratic_0.0.csv};
                \addlegendentry{$0.0$}
                \addplot+[mark=*, mark size=1pt, thick, green] table[x=step, y=value, col sep=comma]{data/couette_abu/loss/bc_left_scale_quadratic_1.0.csv};
                \addlegendentry{$1.0$}
            \end{axis}
        \end{tikzpicture}
        \caption{Левая граница}
        \label{fig:bc_left_scale_quadratic}
    \end{subfigure}
    \hspace{0.5cm}
    \begin{subfigure}[b]{0.4\textwidth}
        \begin{tikzpicture}[scale=0.85]
            \begin{axis}[
                ymode=log,
                legend style={font=\tiny},
                xmin=0,
                xtick distance=4000,
                axis lines=left,
                grid=both
            ]
                \addplot+[mark=*, mark size=1pt, thick, red] table[x=step, y=value, col sep=comma]{data/couette_abu/loss/bc_right_scale_quadratic_0.0.csv};
                \addlegendentry{$0.0$}
                \addplot+[mark=*, mark size=1pt, thick, green] table[x=step, y=value, col sep=comma]{data/couette_abu/loss/bc_right_scale_quadratic_1.0.csv};
                \addlegendentry{$1.0$}
            \end{axis}
        \end{tikzpicture}
        \caption{Правая граница}
        \label{fig:bc_right_scale_quadratic}
    \end{subfigure}
    \caption{Зависимость медианы функции потерь на каждой эпохе при разных коэффициентов для функции активации Quadratic}
    \label{fig:bc_loss_scale_quadratic}
\end{figure}

Аналогичное поведение оптимизаторов можно заметить на графике для нижней
границы (рис. \ref{fig:bc_bottom_scale_quadratic}). На графиках левой
(рис. \ref{fig:bc_left_scale_quadratic}) и правой (рис. \ref{fig:bc_right_scale_quadratic})
границах можно заметить смещение оптимизатора ASGD ближе к Adagrad, что может
свидетельствовать о преобладающем нулевом решении.

%%%%%%%%%%%%%%%%%%%%%%%%%%%%%%%%%%%%%%%%%%%%%%%%%%%%%%%%%%%%%%%%%%%%%%%%%%%%%%

\begin{figure}[htbp]
    \centering
    \begin{subfigure}[b]{0.4\textwidth}
        \centering
        \begin{tikzpicture}[scale=0.85]
            \begin{axis}[
                ymode=log,
                ymax=1e-2,
                legend style={font=\tiny},
                xmin=0,
                xtick distance=4000,
                axis lines=left,
                grid=both
            ]
                \addplot+[mark=*, mark size=1pt, thick, red] table[x=step, y=value, col sep=comma]{data/couette_abu/loss/pde_momentum_x_scale_quadratic_0.0.csv};
                \addlegendentry{$0.0$}
                \addplot+[mark=*, mark size=1pt, thick, green] table[x=step, y=value, col sep=comma]{data/couette_abu/loss/pde_momentum_x_scale_quadratic_1.0.csv};
                \addlegendentry{$1.0$}
            \end{axis}
        \end{tikzpicture}
        \caption{Уравнение для $u_x$}
        \label{fig:pde_ux_scale_quadratic}
    \end{subfigure}
    \hspace{0.5cm}
    \begin{subfigure}[b]{0.4\textwidth}
        \centering
        \begin{tikzpicture}[scale=0.85]
            \begin{axis}[
                ymode=log,
                ymax=1e-2,
                legend style={font=\tiny},
                xmin=0,
                xtick distance=4000,
                axis lines=left,
                grid=both
            ]
                \addplot+[mark=*, mark size=1pt, thick, red] table[x=step, y=value, col sep=comma]{data/couette_abu/loss/pde_momentum_y_scale_quadratic_0.0.csv};
                \addlegendentry{$0.0$}
                \addplot+[mark=*, mark size=1pt, thick, green] table[x=step, y=value, col sep=comma]{data/couette_abu/loss/pde_momentum_y_scale_quadratic_1.0.csv};
                \addlegendentry{$1.0$}
            \end{axis}
        \end{tikzpicture}
        \caption{Уравнение для $u_y$}
        \label{fig:pde_uy_scale_quadratic}
    \end{subfigure}
    \begin{subfigure}[b]{0.7\textwidth}
        \centering
        \begin{tikzpicture}
            \begin{axis}[
                ymode=log,
                ymax=1e-2,
                legend style={font=\small},
                xmin=0,
                xtick distance=1000,
                axis lines=left,
                grid=both,
                width=\textwidth
            ]
                \addplot+[mark=*, mark size=1pt, thick, red] table[x=step, y=value, col sep=comma]{data/couette_abu/loss/pde_continuity_scale_quadratic_0.0.csv};
                \addlegendentry{$0.0$}
                \addplot+[mark=*, mark size=1pt, thick, green] table[x=step, y=value, col sep=comma]{data/couette_abu/loss/pde_continuity_scale_quadratic_1.0.csv};
                \addlegendentry{$1.0$}
            \end{axis}
        \end{tikzpicture}
        \caption{Уравнение непрерывности}
        \label{fig:pde_continuity_scale_quadratic}
    \end{subfigure}
    \caption{Функция потерь для уравнений Навье-Стокса \eqref{eq:navier_stockes} при разных коэффициентов для функции активации Quadratic}
    \label{fig:pde_loss_scale_quadratic}
\end{figure}

Для уравнений Навье-Стокса можно заметить рост функции потерь для оптимизатора
ASGD (рис. \ref{fig:pde_ux_scale_quadratic} и \ref{fig:pde_continuity_scale_quadratic}).
Таким образом происходит процесс поиска глобального минимума. Дело в том,
что функция потерь для верхней границы много больше, чем для уравнений Навье-Стокса
($10^{-0.8}$ против $10^{-2.9}$). Оптимизатор пытается выбраться из локального 
минимума, где решение стремится к нулевому в силу своей корректности с точки
зрения уравнений Навье-Стокса. Что касательно уравнения для скорости $u_y$
(рис. \ref{fig:pde_uy_scale_quadratic}), график оптимизатора ASGD остается
практически неизменным, что опять же соответствует нулевому решению.
Остальные оптимизаторы имеют поведение схожее с поведением на границах.


%%%%%%%%%%%%%%%%%%%%%%%%%%%%%%%%%%%%%%%%%%%%%%%%%%%%%%%%%%%%%%%%%%

\begin{figure}[htbp]
    \centering
    \begin{tikzpicture}
        \begin{axis}[
            width=0.8\textwidth,
            ymode=log,
            xmin=0,
            xtick distance=1000,
            axis lines=left,
            grid=both,
        ]
            \addplot+[mark=*, mark size=1pt, thick, red] table[x=step, y=value, col sep=comma]{data/couette_abu/loss/total_loss_scale_quadratic_0.0.csv};
            \addlegendentry{$0.0$}
            \addplot+[mark=*, mark size=1pt, thick, green] table[x=step, y=value, col sep=comma]{data/couette_abu/loss/total_loss_scale_quadratic_1.0.csv};
            \addlegendentry{$1.0$}
        \end{axis}
    \end{tikzpicture}
    \caption{Итоговая функция потерь при разных коэффициентов для функции активации Quadratic}
    \label{fig:total_loss_scale_quadratic}
\end{figure}

Итого на суммарной функции потерь (рис. \ref{fig:total_loss_scale_quadratic}),
оптимизатор Adam имеет наименьшую функцию потерь.
% \input{images/tikz/results/loss_couette_abu_scale_softplus.tex}
% \begin{figure}[ht]
    \centering
    \begin{subfigure}[b]{0.4\textwidth}
        \begin{tikzpicture}
            \begin{axis}[
                ymode=log,
                % xlabel={Эпоха},
                ylabel={Медиана},
                xmin=0,
                xtick distance=4000,
                axis lines=left,
                grid=both,
                title={Нижняя граница},
                width=\textwidth
            ]
            \addplot+[mark=*, mark size=1pt, thick, red] table[x=step, y=value, col sep=comma]{data/couette_abu/loss/bc_bottom_scale_swish_0.0.csv};
            \addplot+[mark=*, mark size=1pt, thick, green] table[x=step, y=value, col sep=comma]{data/couette_abu/loss/bc_bottom_scale_swish_1.0.csv};
            \end{axis}
        \end{tikzpicture}
        \label{fig:bc_bottom}
    \end{subfigure}
    \hspace{0.5cm}
    \begin{subfigure}[b]{0.4\textwidth}
        \begin{tikzpicture}
            \begin{axis}[
                ymode=log,
                % xlabel={Эпоха},
                % ylabel={Медиана},
                xmin=0,
                xtick distance=4000,
                axis lines=left,
                grid=both,
                title={Верхняя граница},
                width=\textwidth
            ]
            \addplot+[mark=*, mark size=1pt, thick, red] table[x=step, y=value, col sep=comma]{data/couette_abu/loss/bc_top_scale_swish_0.0.csv};
            \addplot+[mark=*, mark size=1pt, thick, green] table[x=step, y=value, col sep=comma]{data/couette_abu/loss/bc_top_scale_swish_1.0.csv};
            \end{axis}
        \end{tikzpicture}
        \label{fig:bc_top}
    \end{subfigure}
    \vspace{0.05cm}
    \begin{subfigure}[b]{0.4\textwidth}
        \begin{tikzpicture}
            \begin{axis}[
                ymode=log,
                xlabel={Эпоха},
                ylabel={Медиана},
                xmin=0,
                xtick distance=4000,
                axis lines=left,
                grid=both,
                title={Левая граница},
                width=\textwidth
            ]
            \addplot+[mark=*, mark size=1pt, thick, red] table[x=step, y=value, col sep=comma]{data/couette_abu/loss/bc_left_scale_swish_0.0.csv};
            \addplot+[mark=*, mark size=1pt, thick, green] table[x=step, y=value, col sep=comma]{data/couette_abu/loss/bc_left_scale_swish_1.0.csv};
            \end{axis}
        \end{tikzpicture}
        \label{fig:bc_left}
    \end{subfigure}
    \hspace{0.5cm}
    \begin{subfigure}[b]{0.4\textwidth}
        \begin{tikzpicture}
            \begin{axis}[
                ymode=log,
                xlabel={Эпоха},
                % ylabel={Медиана},
                xmin=0,
                xtick distance=4000,
                axis lines=left,
                grid=both,
                title={Правая граница},
                width=\textwidth
            ]
            \addplot+[mark=*, mark size=1pt, thick, red] table[x=step, y=value, col sep=comma]{data/couette_abu/loss/bc_right_scale_swish_0.0.csv};
            \addplot+[mark=*, mark size=1pt, thick, green] table[x=step, y=value, col sep=comma]{data/couette_abu/loss/bc_right_scale_swish_1.0.csv};
            \end{axis}
        \end{tikzpicture}
        \label{fig:bc_right}
    \end{subfigure}
    \caption{Зависимость медианы функции потерь на каждой эпохе при разной конфигурации нейронной сети}
    \label{fig:bc_loss_scale_swish}
\end{figure}

%%%%%%%%%%%%%%%%%%%%%%%%%%%%%%%%%%%%%%%%%%%%%%%%%%%%%%%%%%%%%%%%%%%%%%%%%%%%%%

\begin{figure}[htbp]
    \centering
    \begin{subfigure}[b]{0.4\textwidth}
        \begin{tikzpicture}
            \begin{axis}[
                ymode=log,
                ymin=1e-4, ymax=1e-2,
                xlabel={Эпоха},
                ylabel={Медиана},
                xmin=0,
                xtick distance=4000,
                axis lines=left,
                grid=both,
                title={Уравнение для $v_x$},
                width=\textwidth
            ]
            \addplot+[mark=*, mark size=1pt, thick, red] table[x=step, y=value, col sep=comma]{data/couette_abu/loss/pde_momentum_x_scale_swish_0.0.csv};
            \addplot+[mark=*, mark size=1pt, thick, green] table[x=step, y=value, col sep=comma]{data/couette_abu/loss/pde_momentum_x_scale_swish_1.0.csv};
            \end{axis}
        \end{tikzpicture}
    \end{subfigure}
    \hspace{0.5cm}
    \begin{subfigure}[b]{0.4\textwidth}
        \begin{tikzpicture}
            \begin{axis}[
                ymode=log,
                ymin=1e-4, ymax=1e-2,
                xlabel={Эпоха},
                % ylabel={Медиана},
                xmin=0,
                xtick distance=4000,
                axis lines=left,
                grid=both,
                title={Уравнение для $v_y$},
                width=\textwidth
            ]
            \addplot+[mark=*, mark size=1pt, thick, red] table[x=step, y=value, col sep=comma]{data/couette_abu/loss/pde_momentum_y_scale_swish_0.0.csv};
            \addplot+[mark=*, mark size=1pt, thick, green] table[x=step, y=value, col sep=comma]{data/couette_abu/loss/pde_momentum_y_scale_swish_1.0.csv};
            \end{axis}
        \end{tikzpicture}
    \end{subfigure}
    \begin{subfigure}[b]{0.7\textwidth}
        \begin{tikzpicture}
            \begin{axis}[
                ymode=log,
                ymin=1e-4, ymax=1e-2,
                xlabel={Эпоха},
                ylabel={Медиана},
                xmin=0,
                xtick distance=1000,
                axis lines=left,
                grid=both,
                title={Уравнение непрерывности},
                width=\textwidth
            ]
            \addplot+[mark=*, mark size=1pt, thick, red] table[x=step, y=value, col sep=comma]{data/couette_abu/loss/pde_continuity_scale_swish_0.0.csv};
            % \addlegendentry{(32, 64, 32)}
            \addplot+[mark=*, mark size=1pt, thick, green] table[x=step, y=value, col sep=comma]{data/couette_abu/loss/pde_continuity_scale_swish_1.0.csv};
            % \addlegendentry{(64, 32, 64)}
        \end{axis}
        \end{tikzpicture}
    \end{subfigure}
    \caption{Функция потерь для уравнений Навье-Стокса \eqref{eq:navier_stockes} при разной конфигурации нейронной сети}
    \label{fig:pde_loss_scale_swish}
\end{figure}

%%%%%%%%%%%%%%%%%%%%%%%%%%%%%%%%%%%%%%%%%%%%%%%%%%%%%%%%%%%%%%%%%%

\begin{figure}[htbp]
    \centering
    \begin{tikzpicture}
        \begin{axis}[
            width=0.8\textwidth,
            ymode=log,
            xlabel={Эпоха},
            ylabel={Медиана},
            xmin=0,
            xtick distance=1000,
            axis lines=left,
            grid=both,
        ]
            \addplot+[mark=*, mark size=1pt, thick, red] table[x=step, y=value, col sep=comma]{data/couette_abu/loss/total_loss_scale_swish_0.0.csv};
            \addlegendentry{0.0}
            \addplot+[mark=*, mark size=1pt, thick, green] table[x=step, y=value, col sep=comma]{data/couette_abu/loss/total_loss_scale_swish_1.0.csv};
            \addlegendentry{1.0}
        \end{axis}
    \end{tikzpicture}
    \caption{Итоговая функция потерь при разной конфигурации нейронной сети}
    \label{fig:total_loss_scale_swish}
\end{figure}
% \input{images/tikz/results/loss_couette_abu_scale_tanh.tex}

Исходя из полученых результатов можно сделать вывод, что модель со следующими
параметрами (табл. \ref{table:couette_abu_best_params}) является лучшей с точки зрения стабильности.

\begin{table}[h!]
    \centering
    \caption{Значения гиперпараметров у модели с лучшим результатом}
    \begin{tabular}{ |c|c| } 
        \hline
        Функция активации & $\text{ABU}(0, 1, 1, 1)$ \\
        \hline
        Оптимизатор & Adam \\ 
        \hline
        Конфигурация сети & $128-128$ \\ 
        \hline
    \end{tabular}
    \label{table:couette_abu_best_params}
\end{table}

\begin{figure}[ht]
    \includegraphics{data/couette_abu_error_best.png}
    \caption{Лучший результат в процессе кроссвалидации с функцией активации ABU}
    \label{fig:couette_abu_best}
\end{figure}