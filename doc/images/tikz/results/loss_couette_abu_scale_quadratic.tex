\section{Зависимость от фукции активации ABU}

В первую очередь функция активация играет основополагающую роль для
детерминирования поведения внутри домена. Это связано с тем, что
уравнения Навье-Стокса имеют сложную структуру. Если поставленная
задача имеет не нулевое решение, то нахождение верного решения внутри
домена зависит только от функции активации. Это означает, что нет смысла
ориентироваться на функцию потерь на границах.

Как ранее упоминалось, ABU является взвешенной суммой элементарных функций
активации. Рассмотрим влияние каждого слагаемого на функцию потерь.
\subsection{Квадратичная функция}
\begin{figure}[ht]
    \centering
    \begin{subfigure}[b]{0.4\textwidth}
        \begin{tikzpicture}[scale=0.85]
            \begin{axis}[
                ymode=log,
                legend style={font=\tiny},
                xmin=0,
                xtick distance=4000,
                axis lines=left,
                grid=both
            ]            
                \addplot+[mark=*, mark size=1pt, thick, red] table[x=step, y=value, col sep=comma]{data/couette_abu/loss/bc_bottom_scale_quadratic_0.0.csv};
                \addlegendentry{$0.0$}
                \addplot+[mark=*, mark size=1pt, thick, green] table[x=step, y=value, col sep=comma]{data/couette_abu/loss/bc_bottom_scale_quadratic_1.0.csv};
                \addlegendentry{$1.0$}
            \end{axis}
        \end{tikzpicture}
        \caption{Нижняя граница}
        \label{fig:bc_bottom_scale_quadratic}
    \end{subfigure}
    \hspace{0.5cm}
    \begin{subfigure}[b]{0.4\textwidth}
        \begin{tikzpicture}[scale=0.85]
            \begin{axis}[
                ymode=log,
                legend style={font=\tiny},
                xmin=0,
                xtick distance=4000,
                axis lines=left,
                grid=both
            ]
                \addplot+[mark=*, mark size=1pt, thick, red] table[x=step, y=value, col sep=comma]{data/couette_abu/loss/bc_top_scale_quadratic_0.0.csv};
                \addlegendentry{$0.0$}
                \addplot+[mark=*, mark size=1pt, thick, green] table[x=step, y=value, col sep=comma]{data/couette_abu/loss/bc_top_scale_quadratic_1.0.csv};
                \addlegendentry{$1.0$}
            \end{axis}
        \end{tikzpicture}
        \caption{Верхняя граница}
        \label{fig:bc_top_scale_quadratic}
    \end{subfigure}
    \begin{subfigure}[b]{0.4\textwidth}
        \begin{tikzpicture}[scale=0.85]
            \begin{axis}[
                ymode=log,
                legend style={font=\tiny},
                xmin=0,
                xtick distance=4000,
                axis lines=left,
                grid=both
            ]
                \addplot+[mark=*, mark size=1pt, thick, red] table[x=step, y=value, col sep=comma]{data/couette_abu/loss/bc_left_scale_quadratic_0.0.csv};
                \addlegendentry{$0.0$}
                \addplot+[mark=*, mark size=1pt, thick, green] table[x=step, y=value, col sep=comma]{data/couette_abu/loss/bc_left_scale_quadratic_1.0.csv};
                \addlegendentry{$1.0$}
            \end{axis}
        \end{tikzpicture}
        \caption{Левая граница}
        \label{fig:bc_left_scale_quadratic}
    \end{subfigure}
    \hspace{0.5cm}
    \begin{subfigure}[b]{0.4\textwidth}
        \begin{tikzpicture}[scale=0.85]
            \begin{axis}[
                ymode=log,
                legend style={font=\tiny},
                xmin=0,
                xtick distance=4000,
                axis lines=left,
                grid=both
            ]
                \addplot+[mark=*, mark size=1pt, thick, red] table[x=step, y=value, col sep=comma]{data/couette_abu/loss/bc_right_scale_quadratic_0.0.csv};
                \addlegendentry{$0.0$}
                \addplot+[mark=*, mark size=1pt, thick, green] table[x=step, y=value, col sep=comma]{data/couette_abu/loss/bc_right_scale_quadratic_1.0.csv};
                \addlegendentry{$1.0$}
            \end{axis}
        \end{tikzpicture}
        \caption{Правая граница}
        \label{fig:bc_right_scale_quadratic}
    \end{subfigure}
    \caption{Зависимость медианы функции потерь на каждой эпохе при разных коэффициентов для функции активации Quadratic}
    \label{fig:bc_loss_scale_quadratic}
\end{figure}

Аналогичное поведение оптимизаторов можно заметить на графике для нижней
границы (рис. \ref{fig:bc_bottom_scale_quadratic}). На графиках левой
(рис. \ref{fig:bc_left_scale_quadratic}) и правой (рис. \ref{fig:bc_right_scale_quadratic})
границах можно заметить смещение оптимизатора ASGD ближе к Adagrad, что может
свидетельствовать о преобладающем нулевом решении.

%%%%%%%%%%%%%%%%%%%%%%%%%%%%%%%%%%%%%%%%%%%%%%%%%%%%%%%%%%%%%%%%%%%%%%%%%%%%%%

\begin{figure}[htbp]
    \centering
    \begin{subfigure}[b]{0.4\textwidth}
        \centering
        \begin{tikzpicture}[scale=0.85]
            \begin{axis}[
                ymode=log,
                ymax=1e-2,
                legend style={font=\tiny},
                xmin=0,
                xtick distance=4000,
                axis lines=left,
                grid=both
            ]
                \addplot+[mark=*, mark size=1pt, thick, red] table[x=step, y=value, col sep=comma]{data/couette_abu/loss/pde_momentum_x_scale_quadratic_0.0.csv};
                \addlegendentry{$0.0$}
                \addplot+[mark=*, mark size=1pt, thick, green] table[x=step, y=value, col sep=comma]{data/couette_abu/loss/pde_momentum_x_scale_quadratic_1.0.csv};
                \addlegendentry{$1.0$}
            \end{axis}
        \end{tikzpicture}
        \caption{Уравнение для $u_x$}
        \label{fig:pde_ux_scale_quadratic}
    \end{subfigure}
    \hspace{0.5cm}
    \begin{subfigure}[b]{0.4\textwidth}
        \centering
        \begin{tikzpicture}[scale=0.85]
            \begin{axis}[
                ymode=log,
                ymax=1e-2,
                legend style={font=\tiny},
                xmin=0,
                xtick distance=4000,
                axis lines=left,
                grid=both
            ]
                \addplot+[mark=*, mark size=1pt, thick, red] table[x=step, y=value, col sep=comma]{data/couette_abu/loss/pde_momentum_y_scale_quadratic_0.0.csv};
                \addlegendentry{$0.0$}
                \addplot+[mark=*, mark size=1pt, thick, green] table[x=step, y=value, col sep=comma]{data/couette_abu/loss/pde_momentum_y_scale_quadratic_1.0.csv};
                \addlegendentry{$1.0$}
            \end{axis}
        \end{tikzpicture}
        \caption{Уравнение для $u_y$}
        \label{fig:pde_uy_scale_quadratic}
    \end{subfigure}
    \begin{subfigure}[b]{0.7\textwidth}
        \centering
        \begin{tikzpicture}
            \begin{axis}[
                ymode=log,
                ymax=1e-2,
                legend style={font=\small},
                xmin=0,
                xtick distance=1000,
                axis lines=left,
                grid=both,
                width=\textwidth
            ]
                \addplot+[mark=*, mark size=1pt, thick, red] table[x=step, y=value, col sep=comma]{data/couette_abu/loss/pde_continuity_scale_quadratic_0.0.csv};
                \addlegendentry{$0.0$}
                \addplot+[mark=*, mark size=1pt, thick, green] table[x=step, y=value, col sep=comma]{data/couette_abu/loss/pde_continuity_scale_quadratic_1.0.csv};
                \addlegendentry{$1.0$}
            \end{axis}
        \end{tikzpicture}
        \caption{Уравнение непрерывности}
        \label{fig:pde_continuity_scale_quadratic}
    \end{subfigure}
    \caption{Функция потерь для уравнений Навье-Стокса \eqref{eq:navier_stockes} при разных коэффициентов для функции активации Quadratic}
    \label{fig:pde_loss_scale_quadratic}
\end{figure}

Для уравнений Навье-Стокса можно заметить рост функции потерь для оптимизатора
ASGD (рис. \ref{fig:pde_ux_scale_quadratic} и \ref{fig:pde_continuity_scale_quadratic}).
Таким образом происходит процесс поиска глобального минимума. Дело в том,
что функция потерь для верхней границы много больше, чем для уравнений Навье-Стокса
($10^{-0.8}$ против $10^{-2.9}$). Оптимизатор пытается выбраться из локального 
минимума, где решение стремится к нулевому в силу своей корректности с точки
зрения уравнений Навье-Стокса. Что касательно уравнения для скорости $u_y$
(рис. \ref{fig:pde_uy_scale_quadratic}), график оптимизатора ASGD остается
практически неизменным, что опять же соответствует нулевому решению.
Остальные оптимизаторы имеют поведение схожее с поведением на границах.


%%%%%%%%%%%%%%%%%%%%%%%%%%%%%%%%%%%%%%%%%%%%%%%%%%%%%%%%%%%%%%%%%%

\begin{figure}[htbp]
    \centering
    \begin{tikzpicture}
        \begin{axis}[
            width=0.8\textwidth,
            ymode=log,
            xmin=0,
            xtick distance=1000,
            axis lines=left,
            grid=both,
        ]
            \addplot+[mark=*, mark size=1pt, thick, red] table[x=step, y=value, col sep=comma]{data/couette_abu/loss/total_loss_scale_quadratic_0.0.csv};
            \addlegendentry{$0.0$}
            \addplot+[mark=*, mark size=1pt, thick, green] table[x=step, y=value, col sep=comma]{data/couette_abu/loss/total_loss_scale_quadratic_1.0.csv};
            \addlegendentry{$1.0$}
        \end{axis}
    \end{tikzpicture}
    \caption{Итоговая функция потерь при разных коэффициентов для функции активации Quadratic}
    \label{fig:total_loss_scale_quadratic}
\end{figure}

Итого на суммарной функции потерь (рис. \ref{fig:total_loss_scale_quadratic}),
оптимизатор Adam имеет наименьшую функцию потерь.