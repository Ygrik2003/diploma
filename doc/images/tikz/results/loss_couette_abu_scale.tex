\section{Зависимость от фукции активации ABU}

В первую очередь функция активация играет основополагающую роль для
детерминирования поведения внутри домена. Это связано с тем, что
уравнения Навье-Стокса имеют сложную структуру. Если поставленная
задача имеет не нулевое решение, то нахождение верного решения внутри
домена зависит только от функции активации.

Как ранее упоминалось, ABU является взвешенной суммой элементарных функций
активации \eqref{eq:abu_custom}. Рассмотрим влияние каждого слагаемого на
функцию потерь.
\subsection{Квадратичная функция}
Нужно понимать, что введение нелинейности в решение не всегда помогает
найти более точное решение. Наблюдая за функцией потерь
(рис. \ref{fig:pde_loss_scale_quadratic}) можно заметить, что для уравнения
непрерывности (рис. \ref{fig:pde_continuity_scale_quadratic}) и уравнения
компоненты $u_x$ (рис. \ref{fig:pde_ux_scale_quadratic}) поведение 
остается одинаковым, однако для компоненты $u_y$ (рис. \ref{fig:pde_uy_scale_quadratic})
доминирует $\beta_2 = 0.0$.


\begin{figure}[H]
    \centering
    \begin{subfigure}[b]{0.4\textwidth}
        \centering
        \begin{tikzpicture}[scale=0.85]
            \begin{axis}[
                ymode=log,
                ymax=1e-2,
                legend style={font=\tiny},
                xmin=0,
                xtick distance=4000,
                axis lines=left,
                grid=both
            ]
                \addplot+[mark=*, mark size=1pt, thick, red] table[x=step, y=value, col sep=comma]{data/couette_abu/loss/pde_momentum_x_scale_quadratic_0.0.csv};
                \addlegendentry{$0.0$}
                \addplot+[mark=*, mark size=1pt, thick, green] table[x=step, y=value, col sep=comma]{data/couette_abu/loss/pde_momentum_x_scale_quadratic_1.0.csv};
                \addlegendentry{$1.0$}
            \end{axis}
        \end{tikzpicture}
        \caption{Уравнение для $u_x$}
        \label{fig:pde_ux_scale_quadratic}
    \end{subfigure}
    \hspace{0.5cm}
    \begin{subfigure}[b]{0.4\textwidth}
        \centering
        \begin{tikzpicture}[scale=0.85]
            \begin{axis}[
                ymode=log,
                ymax=1e-2,
                legend style={font=\tiny},
                xmin=0,
                xtick distance=4000,
                axis lines=left,
                grid=both
            ]
                \addplot+[mark=*, mark size=1pt, thick, red] table[x=step, y=value, col sep=comma]{data/couette_abu/loss/pde_momentum_y_scale_quadratic_0.0.csv};
                \addlegendentry{$0.0$}
                \addplot+[mark=*, mark size=1pt, thick, green] table[x=step, y=value, col sep=comma]{data/couette_abu/loss/pde_momentum_y_scale_quadratic_1.0.csv};
                \addlegendentry{$1.0$}
            \end{axis}
        \end{tikzpicture}
        \caption{Уравнение для $u_y$}
        \label{fig:pde_uy_scale_quadratic}
    \end{subfigure}
    \begin{subfigure}[b]{0.7\textwidth}
        \centering
        \begin{tikzpicture}[scale=0.85]
            \begin{axis}[
                ymode=log,
                ymax=1e-2,
                legend style={font=\tiny},
                xmin=0,
                xtick distance=2000,
                axis lines=left,
                grid=both
            ]
                \addplot+[mark=*, mark size=1pt, thick, red] table[x=step, y=value, col sep=comma]{data/couette_abu/loss/pde_continuity_scale_quadratic_0.0.csv};
                \addlegendentry{$0.0$}
                \addplot+[mark=*, mark size=1pt, thick, green] table[x=step, y=value, col sep=comma]{data/couette_abu/loss/pde_continuity_scale_quadratic_1.0.csv};
                \addlegendentry{$1.0$}
            \end{axis}
        \end{tikzpicture}
        \caption{Уравнение непрерывности}
        \label{fig:pde_continuity_scale_quadratic}
    \end{subfigure}
    \caption{Функция потерь для уравнений Навье-Стокса \eqref{eq:navier_stockes} при разных коэффициентов для функции активации Quadratic}
    \label{fig:pde_loss_scale_quadratic}
\end{figure}


%%%%%%%%%%%%%%%%%%%%%%%%%%%%%%%%%%%%%%%%%%%%%%%%%%%%%%%%%%%%%%%%%%
В целом данный вывод подтверждается и на графике общей функции потерь
(рис. \ref{fig:pde_loss_scale_quadratic}). Это объясняется линейностью
решения задачи, где отсутствует квадратичная нелинейность.
\begin{figure}[H]
    \centering
    \begin{tikzpicture}
        \begin{axis}[
            width=0.6\textwidth,
            ymode=log,
            xmin=0,
            xtick distance=2000,
            axis lines=left,
            grid=both,
        ]
            \addplot+[mark=*, mark size=1pt, thick, red] table[x=step, y=value, col sep=comma]{data/couette_abu/loss/total_loss_scale_quadratic_0.0.csv};
            \addlegendentry{$0.0$}
            \addplot+[mark=*, mark size=1pt, thick, green] table[x=step, y=value, col sep=comma]{data/couette_abu/loss/total_loss_scale_quadratic_1.0.csv};
            \addlegendentry{$1.0$}
        \end{axis}
    \end{tikzpicture}
    \caption{Итоговая функция потерь при разных коэффициентов для функции активации Quadratic}
    \label{fig:total_loss_scale_quadratic}
\end{figure}

\subsection{Softplus функция}
Данная функция пусть и не линейная, но данная нелинейность проявляется в 
окрестностях нуля, что позволяет посчитать аналитически значение ее производных.
На остальном же промежутке можно ее считать линейной, что вводит линейность в наше решение.
Как видно из графиков для уравнений Навье-Стокса (рис. \ref{fig:pde_loss_scale_softplus}),
при $\beta_1 = 0$ получаемый результат незначительно лучше. 

*налить воды до конца страницы*

\begin{figure}[H]
    \centering
    \begin{subfigure}[b]{0.4\textwidth}
        \centering
        \begin{tikzpicture}[scale=0.85]
            \begin{axis}[
                ymode=log,
                ymax=1e-2,
                legend style={font=\tiny},
                xmin=0,
                xtick distance=4000,
                axis lines=left,
                grid=both
            ]
                \addplot+[mark=*, mark size=1pt, thick, red] table[x=step, y=value, col sep=comma]{data/couette_abu/loss/pde_momentum_x_scale_softplus_0.0.csv};
                \addlegendentry{$0.0$}
                \addplot+[mark=*, mark size=1pt, thick, green] table[x=step, y=value, col sep=comma]{data/couette_abu/loss/pde_momentum_x_scale_softplus_1.0.csv};
                \addlegendentry{$1.0$}
            \end{axis}
        \end{tikzpicture}
        \caption{Уравнение для $u_x$}
        \label{fig:pde_ux_scale_softplus}
    \end{subfigure}
    \hspace{0.5cm}
    \begin{subfigure}[b]{0.4\textwidth}
        \centering
        \begin{tikzpicture}[scale=0.85]
            \begin{axis}[
                ymode=log,
                ymax=1e-2,
                legend style={font=\tiny},
                xmin=0,
                xtick distance=4000,
                axis lines=left,
                grid=both
            ]
                \addplot+[mark=*, mark size=1pt, thick, red] table[x=step, y=value, col sep=comma]{data/couette_abu/loss/pde_momentum_y_scale_softplus_0.0.csv};
                \addlegendentry{$0.0$}
                \addplot+[mark=*, mark size=1pt, thick, green] table[x=step, y=value, col sep=comma]{data/couette_abu/loss/pde_momentum_y_scale_softplus_1.0.csv};
                \addlegendentry{$1.0$}
            \end{axis}
        \end{tikzpicture}
        \caption{Уравнение для $u_y$}
        \label{fig:pde_uy_scale_softplus}
    \end{subfigure}
    \begin{subfigure}[b]{0.7\textwidth}
        \centering
        \begin{tikzpicture}[scale=0.85]
            \begin{axis}[
                ymode=log,
                ymax=1e-2,
                legend style={font=\tiny},
                xmin=0,
                xtick distance=2000,
                axis lines=left,
                grid=both
            ]
                \addplot+[mark=*, mark size=1pt, thick, red] table[x=step, y=value, col sep=comma]{data/couette_abu/loss/pde_continuity_scale_softplus_0.0.csv};
                \addlegendentry{$0.0$}
                \addplot+[mark=*, mark size=1pt, thick, green] table[x=step, y=value, col sep=comma]{data/couette_abu/loss/pde_continuity_scale_softplus_1.0.csv};
                \addlegendentry{$1.0$}
            \end{axis}
        \end{tikzpicture}
        \caption{Уравнение непрерывности}
        \label{fig:pde_continuity_scale_softplus}
    \end{subfigure}
    \caption{Функция потерь для уравнений Навье-Стокса \eqref{eq:navier_stockes} при разных коэффициентов для функции активации Softplus}
    \label{fig:pde_loss_scale_softplus}
\end{figure}

%%%%%%%%%%%%%%%%%%%%%%%%%%%%%%%%%%%%%%%%%%%%%%%%%%%%%%%%%%%%%%%%%%
Однако в силу потерь на границах, на результирующем графике видим обратное:
при $\beta_1 = 1$ результат значительно лучше
(рис. \ref{fig:total_loss_scale_softplus}).

\begin{figure}[H]
    \centering
    \begin{tikzpicture}
        \begin{axis}[
            width=0.6\textwidth,
            ymode=log,
            xmin=0,
            xtick distance=2000,
            axis lines=left,
            grid=both,
        ]
            \addplot+[mark=*, mark size=1pt, thick, red] table[x=step, y=value, col sep=comma]{data/couette_abu/loss/total_loss_scale_softplus_0.0.csv};
            \addlegendentry{$0.0$}
            \addplot+[mark=*, mark size=1pt, thick, green] table[x=step, y=value, col sep=comma]{data/couette_abu/loss/total_loss_scale_softplus_1.0.csv};
            \addlegendentry{$1.0$}
        \end{axis}
    \end{tikzpicture}
    \caption{Итоговая функция потерь при разных коэффициентов для функции активации Softplus}
    \label{fig:total_loss_scale_softplus}
\end{figure}

% \subsection{SiLU функция}
\subsection{SiLU и Tanh функция}
Аналогичный результат присутствует и для функций активации SiLU и Tanh. Одной
из причин такого поведения является маленькая выборка коэффициентов $\beta_i$.
Второй причиной является присутствие шума в виде ранее отсеяных оптимизаторов
и конфигурации слоев. Для более сложных задач следует проводить повторную
кроссвалидацию каждый раз после фильтрации кросспараметров, добавляя больше
параметров для качественного результата.
% \begin{figure}[H]
%     \centering
%     \begin{subfigure}[b]{0.4\textwidth}
%         \centering
%         \begin{tikzpicture}[scale=0.85]
%             \begin{axis}[
%                 ymode=log,
%                 ymax=1e-2,
%                 legend style={font=\tiny},
%                 xmin=0,
%                 xtick distance=4000,
%                 axis lines=left,
%                 grid=both
%             ]
%                 \addplot+[mark=*, mark size=1pt, thick, red] table[x=step, y=value, col sep=comma]{data/couette_abu/loss/pde_momentum_x_scale_swish_0.0.csv};
%                 \addlegendentry{$0.0$}
%                 \addplot+[mark=*, mark size=1pt, thick, green] table[x=step, y=value, col sep=comma]{data/couette_abu/loss/pde_momentum_x_scale_swish_1.0.csv};
%                 \addlegendentry{$1.0$}
%             \end{axis}
%         \end{tikzpicture}
%         \caption{Уравнение для $u_x$}
%         \label{fig:pde_ux_scale_swish}
%     \end{subfigure}
%     \hspace{0.5cm}
%     \begin{subfigure}[b]{0.4\textwidth}
%         \centering
%         \begin{tikzpicture}[scale=0.85]
%             \begin{axis}[
%                 ymode=log,
%                 ymax=1e-2,
%                 legend style={font=\tiny},
%                 xmin=0,
%                 xtick distance=4000,
%                 axis lines=left,
%                 grid=both
%             ]
%                 \addplot+[mark=*, mark size=1pt, thick, red] table[x=step, y=value, col sep=comma]{data/couette_abu/loss/pde_momentum_y_scale_swish_0.0.csv};
%                 \addlegendentry{$0.0$}
%                 \addplot+[mark=*, mark size=1pt, thick, green] table[x=step, y=value, col sep=comma]{data/couette_abu/loss/pde_momentum_y_scale_swish_1.0.csv};
%                 \addlegendentry{$1.0$}
%             \end{axis}
%         \end{tikzpicture}
%         \caption{Уравнение для $u_y$}
%         \label{fig:pde_uy_scale_swish}
%     \end{subfigure}
%     \begin{subfigure}[b]{0.7\textwidth}
%         \centering
%         \begin{tikzpicture}[scale=0.85]
%             \begin{axis}[
%                 ymode=log,
%                 ymax=1e-2,
%                 legend style={font=\tiny},
%                 xmin=0,
%                 xtick distance=2000,
%                 axis lines=left,
%                 grid=both
%             ]
%                 \addplot+[mark=*, mark size=1pt, thick, red] table[x=step, y=value, col sep=comma]{data/couette_abu/loss/pde_continuity_scale_swish_0.0.csv};
%                 \addlegendentry{$0.0$}
%                 \addplot+[mark=*, mark size=1pt, thick, green] table[x=step, y=value, col sep=comma]{data/couette_abu/loss/pde_continuity_scale_swish_1.0.csv};
%                 \addlegendentry{$1.0$}
%             \end{axis}
%         \end{tikzpicture}
%         \caption{Уравнение непрерывности}
%         \label{fig:pde_continuity_scale_swish}
%     \end{subfigure}
%     \caption{Функция потерь для уравнений Навье-Стокса \eqref{eq:navier_stockes} при разных коэффициентов для функции активации SiLU}
%     \label{fig:pde_loss_scale_swish}
% \end{figure}

% %%%%%%%%%%%%%%%%%%%%%%%%%%%%%%%%%%%%%%%%%%%%%%%%%%%%%%%%%%%%%%%%%%

% \begin{figure}[H]
%     \centering
%     \begin{tikzpicture}
%         \begin{axis}[
%             width=0.6\textwidth,
%             ymode=log,
%             xmin=0,
%             xtick distance=2000,
%             axis lines=left,
%             grid=both,
%         ]
%             \addplot+[mark=*, mark size=1pt, thick, red] table[x=step, y=value, col sep=comma]{data/couette_abu/loss/total_loss_scale_swish_0.0.csv};
%             \addlegendentry{$0.0$}
%             \addplot+[mark=*, mark size=1pt, thick, green] table[x=step, y=value, col sep=comma]{data/couette_abu/loss/total_loss_scale_swish_1.0.csv};
%             \addlegendentry{$1.0$}
%         \end{axis}
%     \end{tikzpicture}
%     \caption{Итоговая функция потерь при разных коэффициентов для функции активации SiLU}
%     \label{fig:total_loss_scale_swish}
% \end{figure}


% \subsection{Tanh функция}

% \begin{figure}[H]
%     \centering
%     \begin{subfigure}[b]{0.4\textwidth}
%         \centering
%         \begin{tikzpicture}[scale=0.85]
%             \begin{axis}[
%                 ymode=log,
%                 ymax=1e-2,
%                 legend style={font=\tiny},
%                 xmin=0,
%                 xtick distance=4000,
%                 axis lines=left,
%                 grid=both
%             ]
%                 \addplot+[mark=*, mark size=1pt, thick, red] table[x=step, y=value, col sep=comma]{data/couette_abu/loss/pde_momentum_x_scale_tanh_0.0.csv};
%                 \addlegendentry{$0.0$}
%                 \addplot+[mark=*, mark size=1pt, thick, green] table[x=step, y=value, col sep=comma]{data/couette_abu/loss/pde_momentum_x_scale_tanh_1.0.csv};
%                 \addlegendentry{$1.0$}
%             \end{axis}
%         \end{tikzpicture}
%         \caption{Уравнение для $u_x$}
%         \label{fig:pde_ux_scale_tanh}
%     \end{subfigure}
%     \hspace{0.5cm}
%     \begin{subfigure}[b]{0.4\textwidth}
%         \centering
%         \begin{tikzpicture}[scale=0.85]
%             \begin{axis}[
%                 ymode=log,
%                 ymax=1e-2,
%                 legend style={font=\tiny},
%                 xmin=0,
%                 xtick distance=4000,
%                 axis lines=left,
%                 grid=both
%             ]
%                 \addplot+[mark=*, mark size=1pt, thick, red] table[x=step, y=value, col sep=comma]{data/couette_abu/loss/pde_momentum_y_scale_tanh_0.0.csv};
%                 \addlegendentry{$0.0$}
%                 \addplot+[mark=*, mark size=1pt, thick, green] table[x=step, y=value, col sep=comma]{data/couette_abu/loss/pde_momentum_y_scale_tanh_1.0.csv};
%                 \addlegendentry{$1.0$}
%             \end{axis}
%         \end{tikzpicture}
%         \caption{Уравнение для $u_y$}
%         \label{fig:pde_uy_scale_tanh}
%     \end{subfigure}
%     \begin{subfigure}[b]{0.7\textwidth}
%         \centering
%         \begin{tikzpicture}[scale=0.85]
%             \begin{axis}[
%                 ymode=log,
%                 ymax=1e-2,
%                 legend style={font=\tiny},
%                 xmin=0,
%                 xtick distance=2000,
%                 axis lines=left,
%                 grid=both,
%             ]
%                 \addplot+[mark=*, mark size=1pt, thick, red] table[x=step, y=value, col sep=comma]{data/couette_abu/loss/pde_continuity_scale_tanh_0.0.csv};
%                 \addlegendentry{$0.0$}
%                 \addplot+[mark=*, mark size=1pt, thick, green] table[x=step, y=value, col sep=comma]{data/couette_abu/loss/pde_continuity_scale_tanh_1.0.csv};
%                 \addlegendentry{$1.0$}
%             \end{axis}
%         \end{tikzpicture}
%         \caption{Уравнение непрерывности}
%         \label{fig:pde_continuity_scale_tanh}
%     \end{subfigure}
%     \caption{Функция потерь для уравнений Навье-Стокса \eqref{eq:navier_stockes} при разных коэффициентов для функции активации Tanh}
%     \label{fig:pde_loss_scale_tanh}
% \end{figure}

% %%%%%%%%%%%%%%%%%%%%%%%%%%%%%%%%%%%%%%%%%%%%%%%%%%%%%%%%%%%%%%%%%%

% \begin{figure}[H]
%     \centering
%     \begin{tikzpicture}
%         \begin{axis}[
%             width=0.6\textwidth,
%             ymode=log,
%             xmin=0,
%             xtick distance=2000,
%             axis lines=left,
%             grid=both,
%         ]
%             \addplot+[mark=*, mark size=1pt, thick, red] table[x=step, y=value, col sep=comma]{data/couette_abu/loss/total_loss_scale_tanh_0.0.csv};
%             \addlegendentry{$0.0$}
%             \addplot+[mark=*, mark size=1pt, thick, green] table[x=step, y=value, col sep=comma]{data/couette_abu/loss/total_loss_scale_tanh_1.0.csv};
%             \addlegendentry{$1.0$}
%         \end{axis}
%     \end{tikzpicture}
%     \caption{Итоговая функция потерь при разных коэффициентов для функции активации Tanh}
%     \label{fig:total_loss_scale_tanh}
% \end{figure}
