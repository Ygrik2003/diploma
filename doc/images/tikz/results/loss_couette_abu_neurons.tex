\section{Зависимость от конфигурации нейронной сети}

Сперва стоит подметить, что в данном случае основными компонентами данной задачи
являются функции потерь на нижней и верхней границе в силу условий поставленной
задачи и только потом в учет идут уравнения Навье-Стокса. В большинстве случаев,
отдавая приоритет любой другой компоненте мы получим нулевое решение, что не
удовлетворяет поставленной задачи.

\begin{figure}[ht]
    \centering
    \begin{subfigure}[b]{0.4\textwidth}
        \begin{tikzpicture}[scale=0.85]
            \begin{axis}[
                ymode=log,
                legend style={font=\tiny},
                xmin=0,
                xtick distance=4000,
                axis lines=left,
                grid=both
            ]            
                \addplot+[mark=*, mark size=1pt, thick, red] table[x=step, y=value, col sep=comma]{data/couette_abu/loss/bc_bottom_neurons_(32, 64, 32).csv};
                \addlegendentry{(32, 64, 32)}
                \addplot+[mark=*, mark size=1pt, thick, green] table[x=step, y=value, col sep=comma]{data/couette_abu/loss/bc_bottom_neurons_(64, 32, 64).csv};
                \addlegendentry{(64, 32, 64)}
                \addplot+[mark=*, mark size=1pt, thick, blue] table[x=step, y=value, col sep=comma]{data/couette_abu/loss/bc_bottom_neurons_(64, 64).csv};
                \addlegendentry{(64, 64)}
                \addplot+[mark=*, mark size=1pt, thick, orange] table[x=step, y=value, col sep=comma]{data/couette_abu/loss/bc_bottom_neurons_(128, 128).csv};
                \addlegendentry{(128, 128)}
            \end{axis}
        \end{tikzpicture}
        \caption{Нижняя граница}
        \label{fig:bc_bottom_neurons}
    \end{subfigure}
    \hspace{0.5cm}
    \begin{subfigure}[b]{0.4\textwidth}
        \begin{tikzpicture}[scale=0.85]
            \begin{axis}[
                ymode=log,
                legend style={font=\tiny},
                xmin=0,
                xtick distance=4000,
                axis lines=left,
                grid=both
            ]
                \addplot+[mark=*, mark size=1pt, thick, red] table[x=step, y=value, col sep=comma]{data/couette_abu/loss/bc_top_neurons_(32, 64, 32).csv};
                \addlegendentry{(32, 64, 32)}
                \addplot+[mark=*, mark size=1pt, thick, green] table[x=step, y=value, col sep=comma]{data/couette_abu/loss/bc_top_neurons_(64, 32, 64).csv};
                \addlegendentry{(64, 32, 64)}
                \addplot+[mark=*, mark size=1pt, thick, blue] table[x=step, y=value, col sep=comma]{data/couette_abu/loss/bc_top_neurons_(64, 64).csv};
                \addlegendentry{(64, 64)}
                \addplot+[mark=*, mark size=1pt, thick, orange] table[x=step, y=value, col sep=comma]{data/couette_abu/loss/bc_top_neurons_(128, 128).csv};
                \addlegendentry{(128, 128)}
            \end{axis}
        \end{tikzpicture}
        \caption{Верхняя граница}
        \label{fig:bc_top_neurons}
    \end{subfigure}
    \begin{subfigure}[b]{0.4\textwidth}
        \begin{tikzpicture}[scale=0.85]
            \begin{axis}[
                ymode=log,
                legend style={font=\tiny},
                xmin=0,
                xtick distance=4000,
                axis lines=left,
                grid=both
            ]
                \addplot+[mark=*, mark size=1pt, thick, red] table[x=step, y=value, col sep=comma]{data/couette_abu/loss/bc_left_neurons_(32, 64, 32).csv};
                \addlegendentry{(32, 64, 32)}
                \addplot+[mark=*, mark size=1pt, thick, green] table[x=step, y=value, col sep=comma]{data/couette_abu/loss/bc_left_neurons_(64, 32, 64).csv};
                \addlegendentry{(64, 32, 64)}
                \addplot+[mark=*, mark size=1pt, thick, blue] table[x=step, y=value, col sep=comma]{data/couette_abu/loss/bc_left_neurons_(64, 64).csv};
                \addlegendentry{(64, 64)}
                \addplot+[mark=*, mark size=1pt, thick, orange] table[x=step, y=value, col sep=comma]{data/couette_abu/loss/bc_left_neurons_(128, 128).csv};
                \addlegendentry{(128, 128)}
            \end{axis}
        \end{tikzpicture}
        \caption{Левая граница}
        \label{fig:bc_left_neurons}
    \end{subfigure}
    \hspace{0.5cm}
    \begin{subfigure}[b]{0.4\textwidth}
        \begin{tikzpicture}[scale=0.85]
            \begin{axis}[
                ymode=log,
                legend style={font=\tiny},
                xmin=0,
                xtick distance=4000,
                axis lines=left,
                grid=both
            ]
                \addplot+[mark=*, mark size=1pt, thick, red] table[x=step, y=value, col sep=comma]{data/couette_abu/loss/bc_right_neurons_(32, 64, 32).csv};
                \addlegendentry{(32, 64, 32)}
                \addplot+[mark=*, mark size=1pt, thick, green] table[x=step, y=value, col sep=comma]{data/couette_abu/loss/bc_right_neurons_(64, 32, 64).csv};
                \addlegendentry{(64, 32, 64)}
                \addplot+[mark=*, mark size=1pt, thick, blue] table[x=step, y=value, col sep=comma]{data/couette_abu/loss/bc_right_neurons_(64, 64).csv};
                \addlegendentry{(64, 64)}
                \addplot+[mark=*, mark size=1pt, thick, orange] table[x=step, y=value, col sep=comma]{data/couette_abu/loss/bc_right_neurons_(128, 128).csv};
                \addlegendentry{(128, 128)}
            \end{axis}
        \end{tikzpicture}
        \caption{Правая граница}
        \label{fig:bc_right_neurons}
    \end{subfigure}
    \caption{Зависимость медианы функции потерь на каждой эпохе при разной конфигурации нейронной сети}
    \label{fig:bc_loss_neurons}
\end{figure}

Как можно заметить, на верхней границе (рис. \ref{fig:bc_top_neurons}), стабильнее
всего показывает себя результат при конфигурации нейронной сети $128-128$. В
случае трехслойных конфигураций можно подметить нестабильное поведение, что
может говорить о постоянном попадании в локальные минимумы. Такие конфигурации
нам не подходят в силу комплексности нашей функции потерь, где любая
нестабильность может привести к нулевому решению. Конфигурация $64-64$
на интервале до 4000 эпох не имеет перепадов, но ее результаты хуже,
чем у конфигурации $128-128$. Такая конфигурация может быть полезна, 
если нужно получить не очень точные результаты за короткий промежуток
времени.

%%%%%%%%%%%%%%%%%%%%%%%%%%%%%%%%%%%%%%%%%%%%%%%%%%%%%%%%%%%%%%%%%%%%%%%%%%%%%%

\begin{figure}[H]
    \centering
    \begin{subfigure}[b]{0.4\textwidth}
        \centering
        \begin{tikzpicture}[scale=0.85]
            \begin{axis}[
                ymode=log,
                ymin=1e-4, ymax=1e-2,
                legend style={font=\tiny},
                xmin=0,
                xtick distance=4000,
                axis lines=left,
                grid=both
            ]
                \addplot+[mark=*, mark size=1pt, thick, red] table[x=step, y=value, col sep=comma]{data/couette_abu/loss/pde_momentum_x_neurons_(32, 64, 32).csv};
                \addlegendentry{(32, 64, 32)}
                \addplot+[mark=*, mark size=1pt, thick, green] table[x=step, y=value, col sep=comma]{data/couette_abu/loss/pde_momentum_x_neurons_(64, 32, 64).csv};
                \addlegendentry{(64, 32, 64)}
                \addplot+[mark=*, mark size=1pt, thick, blue] table[x=step, y=value, col sep=comma]{data/couette_abu/loss/pde_momentum_x_neurons_(64, 64).csv};
                \addlegendentry{(64, 64)}
                \addplot+[mark=*, mark size=1pt, thick, orange] table[x=step, y=value, col sep=comma]{data/couette_abu/loss/pde_momentum_x_neurons_(128, 128).csv};
                \addlegendentry{(128, 128)}
            \end{axis}
        \end{tikzpicture}
        \caption{Уравнение для $u_x$}
        \label{fig:pde_ux_neurons}
    \end{subfigure}
    \hspace{0.5cm}
    \begin{subfigure}[b]{0.4\textwidth}
        \centering
        \begin{tikzpicture}[scale=0.85]
            \begin{axis}[
                ymode=log,
                ymin=1e-4, ymax=1e-2,
                legend style={font=\tiny},
                xmin=0,
                xtick distance=4000,
                axis lines=left,
                grid=both
            ]
                \addplot+[mark=*, mark size=1pt, thick, red] table[x=step, y=value, col sep=comma]{data/couette_abu/loss/pde_momentum_y_neurons_(32, 64, 32).csv};
                \addlegendentry{(32, 64, 32)}
                \addplot+[mark=*, mark size=1pt, thick, green] table[x=step, y=value, col sep=comma]{data/couette_abu/loss/pde_momentum_y_neurons_(64, 32, 64).csv};
                \addlegendentry{(64, 32, 64)}
                \addplot+[mark=*, mark size=1pt, thick, blue] table[x=step, y=value, col sep=comma]{data/couette_abu/loss/pde_momentum_y_neurons_(64, 64).csv};
                \addlegendentry{(64, 64)}
                \addplot+[mark=*, mark size=1pt, thick, orange] table[x=step, y=value, col sep=comma]{data/couette_abu/loss/pde_momentum_y_neurons_(128, 128).csv};
                \addlegendentry{(128, 128)}
            \end{axis}
        \end{tikzpicture}
        \caption{Уравнение для $u_y$}
        \label{fig:pde_uy_neurons}
    \end{subfigure}
    \begin{subfigure}[b]{0.4\textwidth}
        \centering
        \begin{tikzpicture}[scale=0.85]
            \begin{axis}[
                ymode=log,
                xmin=0, ymax=1e-2,
                legend style={font=\tiny},
                xtick distance=4000,
                axis lines=left,
                grid=both,
            ]
                \addplot+[mark=*, mark size=1pt, thick, red] table[x=step, y=value, col sep=comma]{data/couette_abu/loss/pde_continuity_neurons_(32, 64, 32).csv};
                \addlegendentry{(32, 64, 32)}
                \addplot+[mark=*, mark size=1pt, thick, green] table[x=step, y=value, col sep=comma]{data/couette_abu/loss/pde_continuity_neurons_(64, 32, 64).csv};
                \addlegendentry{(64, 32, 64)}
                \addplot+[mark=*, mark size=1pt, thick, blue] table[x=step, y=value, col sep=comma]{data/couette_abu/loss/pde_continuity_neurons_(64, 64).csv};
                \addlegendentry{(64, 64)}
                \addplot+[mark=*, mark size=1pt, thick, orange] table[x=step, y=value, col sep=comma]{data/couette_abu/loss/pde_continuity_neurons_(128, 128).csv};
                \addlegendentry{(128, 128)}
            \end{axis}
        \end{tikzpicture}
        \caption{Уравнение непрерывности}
        \label{fig:pde_continuity_neurons}
    \end{subfigure}
    \caption{Функция потерь для уравнений Навье-Стокса \eqref{eq:navier_stockes} при разной конфигурации нейронной сети}
    \label{fig:pde_loss_neurons}
\end{figure}

В добавок можно подметить небольшие флуктуации, возникающие в районе 5000 эпох. Это может свидетельствовать
об переобучении, которого также стоит избегать. В случае задач, где примерный результат известен, не составит
труда отличить верное решение от неверного. При таком исходе флуктуации могут помочь найти еще более
точное решение. Однако если нет возможности оценить правильность результата визуально --- стоит полагаться
только на стабильные участки функции потерь.

Рассмотрим нижнюю границу (рис. \ref{fig:bc_bottom_neurons}). На этой границе в поставленной задаче должно
выполняться равенство нулю всех компонент, что может выполняться в случае нулевого решения. Исключить 
нулевое решение можно оценив поведение уравнений Навье-Стокса для данной модели. В случае, если на верхней
границе функция потерь стремится к нулю и при этом функция потерь уравнения непрерывности (рис. \ref{fig:pde_continuity_neurons})
также стремится к нулю, можно сделать вывод о том, что решение не является нулевым.

Теперь когда мы знаем, что наше решение не нулевое --- осталось проанализировать внутренних переходов скорости,
за которые отвечают функции потери для компонент скорости в уравнениях Навье-Стокса (рис. \ref{fig:pde_ux_neurons}
и \ref{fig:pde_uy_neurons}). Для нашей задачи компонента $u_y$ малозначима в силу ее нулевого значения для всего
решения. Если рассматривать компоненту $v_x$ а также уравнение непрерывности, можно заметить разделение на два
кластера --- трехслойные и двухслойные модели. Пусть трехслойные и показывают лучшие результаты, чем двухслойные,
как мы выяснили ранее, такие модели могут чаще выдавать нулевое решение, что полностью соответствует малым значением
потерь для уравнений Навье-Стокса.


%%%%%%%%%%%%%%%%%%%%%%%%%%%%%%%%%%%%%%%%%%%%%%%%%%%%%%%%%%%%%%%%%%

\begin{figure}[H]
    \centering
    \begin{tikzpicture}
        \begin{axis}[
            width=0.6\textwidth,
            ymode=log,
            xmin=0,
            xtick distance=2000,
            axis lines=left,
            grid=both,
        ]
            \addplot+[mark=*, mark size=1pt, thick, red] table[x=step, y=value, col sep=comma]{data/couette_abu/loss/total_loss_neurons_(32, 64, 32).csv};
            \addlegendentry{(32, 64, 32)}
            \addplot+[mark=*, mark size=1pt, thick, green] table[x=step, y=value, col sep=comma]{data/couette_abu/loss/total_loss_neurons_(64, 32, 64).csv};
            \addlegendentry{(64, 32, 64)}
            \addplot+[mark=*, mark size=1pt, thick, blue] table[x=step, y=value, col sep=comma]{data/couette_abu/loss/total_loss_neurons_(64, 64).csv};
            \addlegendentry{(64, 64)}
            \addplot+[mark=*, mark size=1pt, thick, orange] table[x=step, y=value, col sep=comma]{data/couette_abu/loss/total_loss_neurons_(128, 128).csv};
            \addlegendentry{(128, 128)}
        \end{axis}
    \end{tikzpicture}
    \caption{Итоговая функция потерь при разной конфигурации нейронной сети}
    \label{fig:total_loss_neurons}
\end{figure}

Итого на суммарной функции потерь (рис. \ref{fig:total_loss_neurons}),
конфигурация $128-128$ имеет наименьшую функцию потерь на всех эпохах.