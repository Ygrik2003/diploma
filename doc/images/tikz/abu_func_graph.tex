\begin{figure}[ht]
    \caption{Функции активации на основе REAct}
    \label{fig:abu_func_graph}
    \centering
    \begin{tikzpicture}
        \begin{axis}[
            width=16cm,
            height=10cm,
            xmin=-4, xmax=4,
            ymin=-3, ymax=8,
            axis lines=middle,
            grid=both,
            legend style={at={(1.05,1)},anchor=north west, font=\footnotesize},
            cycle list name=color list,
            samples=200
        ]
        
        % Функция swish(x) = x * sigmoid(x)
        \pgfmathdeclarefunction{swish}{1}{
          \pgfmathparse{#1/(1+exp(-#1))}
        }
        
        % Функция softplus(x) = ln(1 + exp(x))
        \pgfmathdeclarefunction{softplus}{1}{
          \pgfmathparse{ln(1 + exp(#1))}
        }
        
        % Перебор всех комбинаций (a, b, c, d)
        \foreach \a/\b/\c/\d in {0/0/0/0, 0/0/0/1, 0/0/1/0, 0/0/1/1,
                                 0/1/0/0, 0/1/0/1, 0/1/1/0, 0/1/1/1,
                                 1/0/0/0, 1/0/0/1, 1/0/1/0, 1/0/1/1,
                                 1/1/0/0, 1/1/0/1, 1/1/1/0, 1/1/1/1}
        {
            \addplot+[thick] 
            ({x},
             {sin(deg(x)) 
              + (\a == 1 ? tanh(x) : 0)
              + (\b == 1 ? swish(x) : 0)
              + (\c == 1 ? 1/(1 + x^2) : 1)
              + (\d == 1 ? softplus(x) : 0)
             });
            % \addlegendentry{$a=\a,\,b=\b,\,c=\c,\,d=\d$}
        }
        
        \end{axis}
    \end{tikzpicture}
    
\end{figure}