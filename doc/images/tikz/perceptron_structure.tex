\begin{figure}[ht]
    \caption{Структура PINN с использованием функции активации Adaptive Blending Unit\cite{a104fe01d341f235fd80ea98d6a8f35b8110df1d}}
    \label{pinn_structure}
    \centering
    \resizebox{0.8\columnwidth}{!}{
        \begin{tikzpicture}[]
            % Цвета
            \definecolor{inputcolor}{RGB}{173,216,230}
            \definecolor{weightcolor}{RGB}{255,228,181}
            \definecolor{sumcolor}{RGB}{255,182,193}
            \definecolor{actcolor}{RGB}{144,238,144}
            \definecolor{outputcolor}{RGB}{255,255,224}

            % Входы
            \node[draw, circle, fill=inputcolor] (x1) {$x_1$};
            \node[draw, circle, fill=inputcolor, below=0.9cm of x1] (x2) {$x_2$};
            \node[below=0.5cm of x2] (dots) {$\vdots$};
            \node[draw, circle, fill=inputcolor, below=0.5cm of dots] (xn) {$x_n$};

            % Веса
            \node[draw, circle, fill=weightcolor, right=1.2cm of x1] (w1) {$w_{j1}$};
            \node[draw, circle, fill=weightcolor, right=1.2cm of x2] (w2) {$w_{j2}$};
            \node[draw, circle, fill=weightcolor, right=1.2cm of xn] (wn) {$w_{jn}$};

            % Сумматор
            \node[draw, circle, fill=sumcolor, right=2.2cm of w2, minimum size=1.2cm] (sum) {$\sum$};

            % Функция активации
            \node[draw, rectangle, fill=actcolor, right=2.2cm of sum, minimum width=1.2cm, minimum height=1.2cm] (phi) {$\varphi$};

            % Выход
            \node[draw, circle, fill=outputcolor, right=2.2cm of phi] (out) {$o_j$};

            % Стрелки входов к весам
            \draw[->] (x1) -- (w1);
            \draw[->] (x2) -- (w2);
            \draw[->] (xn) -- (wn);

            % Стрелки весов к сумматору
            \draw[->] (w1) -- (sum);
            \draw[->] (w2) -- (sum);
            \draw[->] (wn) -- (sum);

            % Стрелка из сумматора в функцию активации
            \draw[->] (sum) -- (phi);

            % Стрелка из функции активации в выход
            \draw[->] (phi) -- (out);

            % Смещение (порог)
            \node[above=0.5cm of phi] (theta) {$\theta_j$};
            \draw[->, dashed] (theta) -- (phi);

            % Подписи
            \node[above=0.25cm of x1] {Входы};
            \node[above=0.25cm of w1] {Веса};
            \node[above=0.25cm of sum] {Сумма входов};
            \node[below=0.25cm of phi] {Функция активации};
            \node[above=0.25cm of out] {Выход};


        \end{tikzpicture}       
    }
\end{figure}