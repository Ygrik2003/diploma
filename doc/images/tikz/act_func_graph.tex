\begin{figure}[ht]
    \centering
    \begin{tikzpicture}
        \begin{axis}[
            width=14cm, height=10cm,
            grid=both,
            axis y line=middle, axis x line=middle,
            ymin=-1.2, ymax=1.2,
            xmin=-5, xmax=5,
            samples=201,
            legend style={at={(0.5,-0.1)},anchor=north,legend columns=2},
            legend cell align={left},
            title={Активационные функции}
        ]
        
        % Пороговая функция (Heaviside, Step)
        \addplot[thick, blue, domain=-5:5] {x >= 0 ? 1 : 0};
        \addlegendentry{Пороговая (Heaviside)}
        
        % Сигмоида (Logistic Sigmoid)
        \addplot[thick, red, domain=-5:5] {1/(1+exp(-x))};
        \addlegendentry{Сигмоида (Logistic)}
        
        % Гиперболический тангенс (tanh)
        \addplot[thick, green!70!black, domain=-5:5] {tanh(x)};
        \addlegendentry{Гиперб. тангенс (tanh)}
        
        % ReLU
        \addplot[thick, orange, domain=-5:5] {x > 0 ? x : 0};
        \addlegendentry{ReLU}
        
        % Сигмоида сдвинутая (Bipolar Sigmoid)
        \addplot[thick, magenta, domain=-5:5] {(1 - exp(-x))/(1 + exp(-x))};
        \addlegendentry{Сигмоида сдвинутая (Bipolar)}
        
        % Гауссова функция
        \addplot[thick, cyan, domain=-5:5] {exp(-x^2)};
        \addlegendentry{Гауссова функция}
        
        \end{axis}
    \end{tikzpicture}
    \caption{Функции активации на основе Swish}
    \label{fig:act_func_graph}    
\end{figure}