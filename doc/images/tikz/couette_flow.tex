\begin{figure}[ht]
    \caption{Постановка задачи течения Куэтта}
    \label{act_func_graph}
    \centering
    \begin{tikzpicture}[]

        % Размеры
        \def\L{6}   % длина пластин
        \def\H{2}   % расстояние между пластинами
        
        % Пластины
        \draw[thick] (0,0) -- (\L,0); % нижняя пластина
        \draw[thick] (0,\H) -- (\L,\H); % верхняя пластина
        
        % Подписи пластин
        \node[below] at (\L/2,0) {Нижняя пластина (неподвижна)};
        \node[above] at (\L/2,\H) {Верхняя пластина ($U$)};
        
        % Стрелка скорости верхней пластины
        \draw[->,thick,blue] (\L*0.75,\H+0.2) -- (\L*0.75+1,\H+0.2) node[right] {$U$};
        
        % Жидкость между пластинами
        \fill[blue!10] (0,0) rectangle (\L,\H);
        
        % Координаты
        \draw[->] (-0.5,0) -- (-0.5,\H+0.5) node[above] {$y$};
        \draw[->] (-0.5,0) -- (\L+0.5,0) node[right] {$x$};
        
        % Граничные условия
        \draw[->,thick,red] (0.5,0.2) -- (0.5,0.8);
        \node[right,red] at (0.5,0.5) {$u=0$};
        
        \draw[->,thick,red] (0.5,\H-0.2) -- (0.5,\H-0.8);
        \node[right,red] at (0.5,\H-0.5) {$u=U$};
        
        % Подпись слоя жидкости
        \node at (\L/2,\H/2) {Жидкость};
        
        \end{tikzpicture}
    
\end{figure}