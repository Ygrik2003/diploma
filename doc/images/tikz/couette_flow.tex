\begin{figure}[ht]
    \caption{Постановка задачи течения Куэтта}
    \label{act_func_graph}
    \centering
    \begin{tikzpicture}[scale=1.5]

        % Оси координат
        \draw[->] (0,0) -- (7,0) node[below] {$x$};
        \draw[->] (0,0) -- (0,3) node[left] {$y$};
    
        % Пластины
        \draw[thick] (0,0.5) -- (6.5,0.5) node[midway, below] {Нижняя пластина ($u=0$)};
        \draw[thick] (0,2.5) -- (6.5,2.5) node[midway, above] {Верхняя пластина ($u=U_0$)};
    
        % Расстояние между пластинами
        \draw[decorate, decoration={brace, amplitude=5pt, mirror}] (6.7,0.5) -- (6.7,2.5) 
            node[midway, right=5pt] {$h$};
    
        % Течение Куэтта (линейный профиль скорости)
        \foreach \y in {0.6,0.8,...,2.4} {
            \pgfmathsetmacro{\arrowlength}{2*(\y-0.5)/2} % Вычисляем длину стрелки
            \draw[blue, ->] (0,\y) -- (0+\arrowlength,\y); % Рисуем стрелку
        }
    
        % Подписи
        \node at (3,1.5) {$\vec{u}(y) = U_0 \frac{y}{h}$};
        
    \end{tikzpicture}
\end{figure}