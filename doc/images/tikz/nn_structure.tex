\begin{figure}[ht]
    \centering
    \resizebox{0.8\columnwidth}{!}{
        \begin{tikzpicture}[node distance=1.2cm, thick]

            % Цвета нод
            \definecolor{inputcolor}{RGB}{255,230,230}   % светло-розовый
            \definecolor{hiddencolor}{RGB}{200,220,255}  % светло-голубой
            \definecolor{outputcolor}{RGB}{220,255,220}  % светло-зелёный
            
            % Входной слой
            \foreach \i in {1,2,3}
                \node[draw, circle, fill=inputcolor, minimum size=1cm] (I\i) at (0,1.8-\i*1.2) {};
            
            % Скрытый слой 1
            \foreach \i in {1,2,3,4}
                \node[draw, circle, fill=hiddencolor, minimum size=1cm] (Hone\i) at (2.5,2.4-\i*1.2) {};
            
            % Скрытый слой 2
            \foreach \i in {1,2,3,4}
                \node[draw, circle, fill=hiddencolor, minimum size=1cm] (Htwo\i) at (5.5,2.4-\i*1.2) {};
            
            % Выходной слой
            \node[draw, circle, fill=outputcolor, minimum size=1cm] (O) at (8.2,-0.6) {};
            
            % Связи: входной -> скрытый 1
            \foreach \i in {1,2,3}
                \foreach \j in {1,2,3,4}
                    \draw[->] (I\i) -- (Hone\j);
            
            % Связи: скрытый 1 -> скрытый 2
            \foreach \i in {1,2,3,4}
                \foreach \j in {1,2,3,4}
                    \draw[->] (Hone\i) -- (Htwo\j);
            
            % Связи: скрытый 2 -> выходной
            \foreach \i in {1,2,3,4}
                \draw[->] (Htwo\i) -- (O);
            
            % Подписи слоёв
            \node[align=center] at (-0.2,2.8) {Входной\\слой};
            \node[align=center] at (2.5,2.8) {Скрытый\\слой 1};
            \node[align=center] at (5.5,2.8) {Скрытый\\слой 2};
            \node[align=center] at (8.2,2.8) {Выходной\\слой};
        \end{tikzpicture}       
    }
    \caption{Структура PINN с использованием функции активации Adaptive Blending Unit\cite{a104fe01d341f235fd80ea98d6a8f35b8110df1d}}
    \label{fig:nn_structure}
\end{figure}