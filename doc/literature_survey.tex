\chapter{Теоретические основы физически-инфромированных нейронных сетей}

В последние годы наблюдается значительный прорыв в области численного моделирования движения жидкостей,
что связано не только с развитием традиционных методов, таких как методы конечных разностей, конечных
объемов и элементов, но и с появлением принципиально новых подходов на основе машинного обучения.
Физически информированные нейронные сети представляют собой
инновационный метод, объединяющий глубокое обучение с фундаментальными физическими законами для
моделирования сложных гидродинамических процессов \cite{raissi2019physics}.

\section{Концепция и принципы работы}
Физически информированные нейронные сети представляют собой разновидность искусственных нейронных сетей,
которые интегрируют знания о физических законах непосредственно в процесс обучения. Появившись в 2017
году \cite{raissi2017physics}, данный метод позволяет моделировать физические процессы с учетом уравнений,
описывающих эти процессы, путем интеграции дифференциальных уравнений и граничных условий в структуру
функции потерь.

Функция потерь (целевая функция) формализует задачу оптимизации параметров нейронной сети, выступая
дифференцируемой мерой расхождения между предсказанными значениями модели и эталонными данными.
На каждой итерации обучения (эпохе) вычисляется усреднённое значение функции потерь по тренировочному
набору данных. Обратное распространение ошибки реализует эффективное вычисление
градиентов функции потерь по параметрам модели через цепное правило дифференцирования. 
Экспериментально доказано, что комбинация дифференцируемых функций потерь
с обратным распространением ошибки обеспечивает сходимость к локальным минимумам даже для сетей с миллионами
параметров \cite{Goodfellow-et-al-2016}.

Основная идея физически-информированных нейронных сетей заключается в конструировании функции потерь,
которая включает не только отклонения
от обучающих данных, но и степень нарушения физических законов \cite{karniadakis2021physics}. Это
позволяет нейронной сети «изучать» физические принципы, лежащие в основе моделируемых процессов. При
обучении нейронной сети для моделирования течения жидкости используется автоматическое дифференцирование для
вычисления требуемых производных в уравнениях, что значительно упрощает решение сложных дифференциальных
уравнений в частных производных. Однако остается возможность и ручного численного вычисления производных
для более сложных задач.

\section{Математическая формулировка}
Моделирование движения жидкости традиционно основывается на решении уравнений Навье-Стокса, которые
представляют собой систему дифференциальных уравнений в частных производных, описывающих законы сохранения
массы, импульса и энергии для движущейся жидкости \cite{batchelor2000introduction}. В контексте PINN эти
уравнения включаются в функцию потерь нейронной сети, которая минимизируется в процессе
обучения \cite{yang2019adversarial}.

\begin{equation}
    \begin{cases}
    \nabla \cdot \mathbf{u} = 0 \\
    \frac{\partial \mathbf{u}}{\partial t} + (\mathbf{u} \cdot \nabla) \mathbf{u} = -\nabla p + \frac{1}{Re} \nabla^2 \mathbf{u} + \mathbf{f}
    \end{cases}
    \label{eq:navier_stockes}
\end{equation}
где $p$ — давление, $Re$ — число Рейнольдса, определяемое как $Re = \frac{UL}{\nu}$, где $U$ —
характерная скорость, $L$ — характерная длина, $\nu$ — кинематическая вязкость, $\mathbf{f}$ —
внешняя сила (например, гравитация), $\mathbf{u}$ — вектор скорости.

Для случая вязкой несжимаемой жидкости уравнения Навье-Стокса в переменных скорость-давление формулируются
как система уравнений, включающая уравнения движения и уравнение неразрывности. PINN аппроксимирует
решение этих уравнений, обучаясь на данных и одновременно удовлетворяя физическим ограничениям, заданным
этими уравнениями \cite{jin2021nsfnets}.

\section{Функции активации в PINN}

Традиционные функции активации используются без адаптации под специфику задачи. К наиболее распространённым
относятся функции $tanh$, $sin$ и $sigmoid$, которые часто применяются для нейронных сетей благодаря их гладкости
и возможности вычисления производных высших порядков \cite{0d752c79fb816703274a3d37f85a85689a2a9405}
\cite{Sutfeld2018-io}. В то же время функции $ReLU$ и их модификации не подходят для физически-информированных нейронных сетей из-за
разрывов вторых производных \cite{fe520ccac2a6bd50f75a4a34022fe54116871013}. Для задач с быстрым затуханием
эффективны экспоненциальные функции, такие как $exp$ \cite{7fcd4b3c875d8e41eb0c184aa1a42bf4c8906d61}.

Производительность таких функций сильно зависит от свойств решаемого дифференциального уравнения в частных
производных (PDE).  Таким образом, $sin$ лучше подходит для периодических систем (например, уравнение
Пуассона), а $exp$ — для задач с экспоненциальным затуханием \cite{fe520ccac2a6bd50f75a4a34022fe54116871013}.

Адаптивные функции активации содержат обучаемые параметры, которые подстраиваются под уравнения. Примером
является функция \textbf{REAct (Rational Exponential Activation)}
\cite{0d752c79fb816703274a3d37f85a85689a2a9405}:
\begin{equation}
\text{REAct}(x) = \frac{1 - e^{ax + b}}{1 + e^{cx + d}},
\label{eq:react}
\end{equation}
где $a$, $b$, $c$, $d$ — обучаемые параметры, обобщающие $\tanh$. 

Преимущества REAct включают снижение среднеквадратической ошибки (MSE) и устойчивость к шуму
\cite{0d752c79fb816703274a3d37f85a85689a2a9405}.

Также исследованы адаптивные наклоны \cite{7fcd4b3c875d8e41eb0c184aa1a42bf4c8906d61}, где вводится параметр
$\beta$: $f(x) = \sigma(\beta x)$, оптимизируемый для ускорения сходимости.

Для повышения гибкости также используются комбинации базовых функций активации. Одним из таких методов является
\textbf{ABU-PINN (Adaptive Blending Units)} (рис. \ref{pinn_structure}) \cite{Sutfeld2018-io}\cite{7fcd4b3c875d8e41eb0c184aa1a42bf4c8906d61}:
\begin{equation}
    f(x) = \sum_{i=1}^N G(\alpha_i) \sigma_i(\beta_i x),
    \label{eq:abu}
\end{equation}
где $G$ — функция взвешивания (например, softmax), а $\sigma_i$ — набор кандидатных функций, таких как
$\sin$, $\tanh$, $\mathrm{GELU}$, $\mathrm{Swish}$. 

Основные особенности метода включают набор кандидатов, состоящий из функций с различными свойствами, а
также добавление элементарных функций, таких как $\mathrm{exp}$ и $\mathrm{sin}$, на основе априорных знаний о PDE.
Для уравнений, обладающих периодичностью и затуханием, ABU-PINN автоматически усиливает функции
$\sin$ и $\exp$, что позволяет более эффективно учитывать особенности таких уравнений
\cite{7fcd4b3c875d8e41eb0c184aa1a42bf4c8906d61}.

\begin{figure}[ht]
    \centering
    \resizebox{0.8\columnwidth}{!}{
        \begin{tikzpicture}[
            % Define node styles
            neuron/.style={ellipse, draw, fill=brown!80, minimum width=1.3cm, minimum height=0.8cm},
            input/.style={circle, draw, fill=blue!70, minimum size=0.8cm},
            output/.style={circle, draw, fill=red!70, minimum size=0.8cm},
            pde/.style={circle, draw, fill=yellow!70, minimum size=1cm},
            operation/.style={rectangle, draw, fill=green!70, rounded corners, minimum width=4cm, minimum height=1cm},
            loss/.style={ellipse, draw, fill=cyan!60, minimum width=2cm, minimum height=1cm},
            decision/.style={diamond, draw, fill=yellow!40, text width=2cm, align=center},
            arrow/.style={thick, ->, >=stealth},
            box/.style={rectangle, draw, dashed, rounded corners, inner sep=10pt}
        ]

        % Neural Network part
        \coordinate (nn_top_left) at (-8,4);
        \coordinate (nn_bottom_right) at (1,-4);
        \node[box, fit={(nn_top_left) (nn_bottom_right)}] (nn) {};
        \node[above] at (nn.north) {\Large \textbf{NN}$(w, \boldsymbol{a})$};

        % Input layer
        \node[input] (x) at (-7,1) {$\boldsymbol{x}$};
        \node[input] (t) at (-7,-1) {$t$};

        % Hidden layers
        \node[neuron] (h11) at (-5,3) {$\sigma(a_1^1)$};
        \node[neuron] (h12) at (-5,1) {$\sigma(a_2^1)$};
        \node[neuron] (h13) at (-5,-1) {$\sigma(a_3^1)$};
        \node[neuron] (h14) at (-5,-3) {$\sigma(a_4^1)$};

        \node[neuron] (h21) at (-2,3) {$\sigma(a_1^2)$};
        \node[neuron] (h22) at (-2,1) {$\sigma(a_2^2)$};
        \node[neuron] (h23) at (-2,-1) {$\sigma(a_3^2)$};
        \node[neuron] (h24) at (-2,-3) {$\sigma(a_4^2)$};

        % Output
        \node[output] (u) at (0,0) {$y$};

        % Connect input to first hidden layer
        \draw[arrow] (x) -- (h11);
        \draw[arrow] (x) -- (h12);
        \draw[arrow] (x) -- (h13);
        \draw[arrow] (x) -- (h14);
        \draw[arrow] (t) -- (h11);
        \draw[arrow] (t) -- (h12);
        \draw[arrow] (t) -- (h13);
        \draw[arrow] (t) -- (h14);

        % Connect first hidden layer to second hidden layer
        \foreach \i in {1,2,3,4}
            \foreach \j in {1,2,3,4}
                \draw[arrow] (h1\i) -- (h2\j);

        % Connect second hidden layer to output
        \draw[arrow] (h21) -- (u);
        \draw[arrow] (h22) -- (u);
        \draw[arrow] (h23) -- (u);
        \draw[arrow] (h24) -- (u);

        % PDE part
        \coordinate (pde_top_left) at (2,4);
        \coordinate (pde_bottom_right) at (9,-4);
        \node[box, fit={(pde_top_left) (pde_bottom_right)}] (pde) {};
        \node[above] at (pde.north) {\Large \textbf{PDE}$(y)$};

        \node[pde] (f) at (3,2) {$f(\cdot)$};
        \node[pde] (dt) at (3,0) {$\partial$};
        \node[pde] (nabla) at (3,-2) {$\nabla \cdot$};

        \node[operation] (residual) at (6.2,0) {$\mathcal{F}:= f(u, x_i, \frac{\partial{u}}{\partial{x_i}}, \frac{\partial^2{u}}{\partial{x_i^2}})$};

        % Connect PDE components
        \draw[arrow] (u) -- (f);
        \draw[arrow] (u) -- (dt);
        \draw[arrow] (u) -- (nabla);
        \draw[arrow] (f) -- (residual);
        \draw[arrow] (dt) -- (residual);
        \draw[arrow] (nabla) -- (residual);

        % Loss function and training control
        \node[loss] (loss) at (1,-6) {$\mathcal{L}$};

        % Adaptive parameter node
        \node[circle, draw, fill=purple!70, minimum size=0.8cm] (adapt) at (-1,-6) {$a_i^k$};

        % Decision diamond
        \node[decision] (decision) at (-5,-7) {$< \epsilon$};

        % Final state
        \node (done) at (-5,-10) {Успех};

        % Connect other components
        \draw[arrow] (u) to[out=-90, in=45] (loss);
        \draw[arrow] (residual) to[out=-90, in=0] (loss);
        \draw[arrow] (loss) to[out=-135, in=0] (decision);
        \draw[arrow] (decision) -- node[left] {Y} (done);
        \draw[arrow] (loss) to[out=180, in=0] (adapt);
        % \draw[arrow] (loss) -- (decision);

        % \draw[arrow] (adapt) -- node[above] (loss);
        % \draw[arrow] (nn) -- node[above] (adapt);

        \draw[arrow] (adapt) to[out=180, in=-90] (nn.south);
        \draw[arrow] (decision) to[out=90, in=-90] ($(nn.south) - (1.5,0)$);

        \end{tikzpicture}       
    }
    \caption{Структура PINN с использованием функции активации Adaptive Blending Unit\cite{a104fe01d341f235fd80ea98d6a8f35b8110df1d}}
    \label{pinn_structure}
\end{figure}

Другие подходы могут также включать аппроксимацию полиномами Тейлора ($f(x) = \sum a_i x^i$) или
рациональные функции Паде ($f(x) = \frac{P(x)}{Q(x)}$).
Проблема, с которой можно столкнуться, это неустойчивость при высоких производных и сложность
интерпретации \cite{7fcd4b3c875d8e41eb0c184aa1a42bf4c8906d61}.

Подбор функций активации может выполняться методом проб и ошибок с учётом свойств PDE. Для гладких решений
предпочтение отдается функциям с непрерывными производными, таким как $\tanh$ и $\sin$. В случае периодических
систем используются функции $\sin$ и $\cos$, а для задач с затуханием применяются $\exp$ и $\tanh$.

% [REAct](https://github.com/srvmishra/REAct) и [ABU-PINN](https://github.com/LeapLabTHU/AdaAFforPINNs).

\section{Преимущества использования PINN в моделировании жидкости}
Использование PINN для моделирования движения жидкости обладает рядом существенных преимуществ по сравнению
с традиционными численными методами \cite{cuomo2022scientific}. Прежде всего, PINN интегрирует физическую
интуицию и уравнения в модель, что повышает точность решения. Кроме того, этот метод не требует создания
расчетной сетки, что упрощает моделирование для задач со сложной геометрией \cite{cai2021physics}.

Одним из ключевых преимуществ физически-информированных нейроных сетей является их гибкость –-- они подходят для решения любых задач и
не требуют полных и точных данных, что особенно ценно в ситуациях с ограниченной
информацией \cite{karniadakis2021physics}. Физически информированные нейронные сети также способны решать
сложные задачи, включая моделирование турбулентных течений, что делает их мощным инструментом в современной
гидродинамике \cite{mao2020physics}.

Нейронные сети существенно сокращают время расчета по сравнению с традиционными численными
методами \cite{jagtap2020conservative}. Это особенно важно для сложных задач, требующих значительных вычислительных
ресурсов. В целом такой подход может достигать аналогичной точности с традиционными методами CFD, но с существенно меньшими
вычислительными затратами, что может привести к сокращению времени вычислений. \cite{Tommaso2024pinn}.

Развитие технологий GPU-ускорения дополнительно повышает эффективность метода. Современные реализации PINN, такие
как библиотека NeuralPDE.jl \cite{neuralpde2023}, позволяют использовать графические процессоры для значительного
ускорения вычислений, что делает этот метод еще более привлекательным для решения ресурсоемких задач гидродинамики.

\section{Сравнение PINN с традиционными методами численного моделирования}
Традиционные методы численного моделирования, такие как метод конечных разностей, требуют разбиения исследуемой области
на множество маленьких фрагментов (создания расчетной сетки), что может быть сложной и неточной операцией, особенно в
случае нестандартной формы области \cite{ferziger2019computational}. PINN, в отличие от традиционных методов, не требуют
создания сетки, что упрощает моделирование для задач со сложной геометрией \cite{karniadakis2021physics}.

В задачах, где работает численное моделирование, нейросеть может быстрее получить примерное
решение \cite{kochkov2021machine}. Однако особенно ценными PINN становятся в задачах с некорректной классической постановкой,
где традиционные методы могут давать неустойчивые или неединственные решения. В таких случаях PINN выступает с численными
методами на равных или даже превосходит их, выбирая из всех возможных решений наиболее физически обоснованное
\cite{yang2019adversarial}.

Обученная сеть PINN может использоваться для прогнозирования значений на имитационных сетках различного разрешения без
необходимости переобучения, что значительно упрощает практическое применение метода \cite{raissi2019physics}. Кроме того,
PINN позволяют использовать автоматическое дифференцирование для вычисления требуемых производных в уравнениях с частными
производными, что по оценкам превосходит численное или символьное дифференцирование \cite{baydin2018automatic}.

Следует отметить, что PINN особенно эффективны в ситуациях, когда данные ограничены или разрежены, что часто встречается
в реальных инженерных и научных задачах \cite{zhu2019physics}. Это делает их ценным инструментом для исследователей и
инженеров, работающих с реальными гидродинамическими системами.

\section{Проблемы и ограничения PINN в гидродинамике}
Несмотря на значительные преимущества, методология PINN находится на ранней стадии развития, и многие аспекты еще требуют
изучения \cite{cuomo2022scientific}. Одним из ключевых вызовов является определение наиболее эффективных инструментов
и стратегий использования PINN в зависимости от типа уравнений и конкретной задачи. Существует неопределенность в отношении
наилучших конфигураций нейросетей для различных ситуаций, и универсального подхода к выбору оптимальных методов и техник
для различных уравнений и задач пока не существует \cite{krishnapriyan2021characterizing}.

Отсутствие единой системы рекомендаций или стандартизированных процедур затрудняет широкое применение PINN в практических
задачах. Это создает необходимость в дополнительных исследованиях и разработке методических рекомендаций для эффективного
использования PINN в гидродинамике \cite{wang2022respecting}.

Технические ограничения PINN связаны прежде всего с вычислительными требованиями и сложностью обучения нейронных сетей для
решения дифференциальных уравнений высокого порядка \cite{wang2021understanding}. Хотя GPU-ускорение значительно повышает
эффективность вычислений, для сложных трехмерных задач с многофазными течениями или химическими реакциями требуются
значительные вычислительные ресурсы.

Еще одним техническим ограничением является отсутствие единого программного решения, охватывающего все разнообразие
необходимых инструментов и возможностей для работы с PINN в гидродинамике \cite{cuomo2022scientific}. Это стимулирует
разработку специализированных программных библиотек и инструментов, таких как NeuralPDE.jl \cite{neuralpde2023}, но также
создает фрагментацию в экосистеме PINN.

\section{Перспективы развития PINN для моделирования жидкости}
С ростом популярности метода PINN наблюдается увеличение числа научных публикаций, исследующих различные аспекты применения
этого метода в гидродинамике \cite{karniadakis2021physics}. Перспективными направлениями исследований являются разработка
оптимальных архитектур нейронных сетей для конкретных типов задач, методы регуляризации для повышения устойчивости решений,
а также интеграция PINN с другими методами машинного обучения и традиционными численными методами \cite{jagtap2022physics}.

Особый интерес представляет применение PINN для моделирования турбулентных течений, многофазных систем и течений с химическими
реакциями, что расширит область применения этого метода в промышленности и научных исследованиях \cite{sun2020surrogate}.

Развитие технологий GPU-ускорения и специализированных библиотек, таких как NeuralPDE.jl \cite{neuralpde2023}, делает PINN
более доступными и эффективными для широкого круга исследователей и инженеров. Интеграция PINN в существующие системы
автоматизированного проектирования и анализа (CAD/CAE) может значительно упростить их практическое применение в
промышленности \cite{cuomo2022scientific}.

Разработка собственного программного обеспечения для работы с PINN, как упоминается в \cite{lu2021deepxde}, является важным
технологическим трендом, который может способствовать стандартизации и упрощению использования этого метода в гидродинамике.
