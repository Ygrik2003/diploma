\chapter*{Введение}

В последние годы методы машинного обучения и искусственного интеллекта находят всё
более широкое применение в различных областях науки и техники. Одним из перспективных
направлений является использование нейронных сетей для решения дифференциальных
уравнений и их систем, которые лежат в основе математического моделирования физических,
химических, биологических и других процессов. В этом контексте особый интерес представляют
физически-информированные нейронные сети (Physics-Informed Neural Networks, PINN) ---
гибридный подход, объединяющий мощь нейронных сетей с фундаментальными законами физики,
выраженными в виде дифференциальных уравнений.

Целью работы являлась разработка и исследование модели движения жидкости на основе
физически-информированных нейронных сетей для повышения точности и эффективности
численного моделирования гидродинамических процессов с учётом физических законов и
ограничений. В рамках исследования были изучены теоретические основы и математическая
формулировка физически-информированных нейронных сетей применительно к задачам
гидродинамики, проведён сравнительный анализ преимуществ и ограничений PINN по
сравнению с традиционными численными методами моделирования течений жидкости. Также
была разработана архитектура нейронной сети с использованием специализированных функций
активации и методов оптимизации, адаптированных для решения уравнений движения жидкости.
Реализована вычислительная модель, интегрирующая граничные условия и физические ограничения,
характерные для рассматриваемых гидродинамических систем. Проведены обучение и валидация
модели на тестовых примерах с использованием эталонных аналитических решений и
экспериментальных данных. В итоге определены перспективные направления развития
методологии физически-информированных нейронных сетей в контексте гидродинамического
моделирования.

Объектом исследования в данной работе является моделирование движения жидкости.
Предмет исследования — применение физически-информированных нейронных сетей
для повышения точности и эффективности численного моделирования гидродинамических
процессов с учётом физических законов и ограничений.

Научная новизна физически информированных нейронных сетей заключается в интеграции
физических законов, описываемых дифференциальными уравнениями, непосредственно в
архитектуру нейронных сетей. Это позволяет моделям не только обучаться на данных,
но и строго соблюдать фундаментальные принципы физики, что значительно повышает
точность и устойчивость решений сложных задач гидродинамики и других областей.

Практическая значимость заключается в отсутствии необходимости в большом количестве
экспериментальных данных, в быстром воспроизведении решения, а также минимизации
объема информации, необходимом для воспроизведения решения с большим количеством
параметров. Также такой подход позволяет легко использовать параллельные вычисления
как на CPU, так и на GPU. С точки зрения гидродинамики это может быть полезно в игровой индустрии
для уменьшения нагрузки на устройство пользователя. Практически все новые мобильные
устройства содержат NPU, что позволяет на аппаратном уровне ускорить и облегчить
воспроизведение результатов обучения. Такой подход может найти применение и в 
киноиндустрии: «запекание» с возможностью подробной параметризации модели позволит
ускорить расчет для финальной сцены, а также ускоряют и упрощают процесс повторных съёмок.
В данном контексте следует отметить, что физически информированные нейронные сети способны
генерировать визуально достоверные решения, несмотря на возможные отклонения от полного
физического соответствия.
