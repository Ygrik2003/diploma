\chapter*{Реферат}
\addcontentsline{toc}{chapter}{Реферат}
Общий общий объем работы $40$ страниц, $21$ рисунок, $4$ таблицы, $33$ источника.

Физически-информированные нейронные сети, уравнения Навье-Стокса,
компьютерное моделирование, безсеточный метод, течене Куэтта.

В последние годы методы машинного обучения и искусственного интеллекта активно применяются
для решения дифференциальных уравнений, лежащих в основе моделирования физических процессов.
Особый интерес представляют физически-информированные нейронные сети (PINN), которые
объединяют возможности нейросетей с фундаментальными физическими законами. Целью исследования
была разработка модели движения жидкости на основе PINN для решения системы уравнений Навье-Стокса.
В работе изучены теоретические основы PINN, проведён их
сравнительный анализ с традиционными численными методами, разработана специализированная
архитектура нейросети с адаптированными функциями активации и методами оптимизации.
Реализованная модель интегрирует граничные условия и физические ограничения, прошла обучение
и валидацию на тестовых примерах с использованием аналитических решений.
Научная новизна подхода заключается в прямой интеграции физических законов в архитектуру
нейросети, что обеспечивает соблюдение фундаментальных принципов физики. Практические
преимущества включают снижение зависимости от экспериментальных данных, высокую скорость
вычислений и эффективное использование вычислительных ресурсов (CPU, GPU, NPU). 
Метод способен генерировать визуально достоверные решения даже при неполном физическом соответствии.

% English Translation
\chapter*{Abstract}
\addcontentsline{toc}{chapter}{Abstract}
The total volume of the work is $40$ pages, $21$ figures, $4$ tables, $33$ sources.

Physics-informed neural networks, Navier-Stokes equations, 
computer simulation, meshless method, Couette flow.

In recent years, machine learning and artificial intelligence methods have been actively used 
to solve differential equations underlying the modeling of physical processes. 
Particular interest is drawn to physics-informed neural networks (PINNs), which 
combine the capabilities of neural networks with fundamental physical laws. The research aimed 
to develop a fluid motion model based on PINNs for solving the Navier-Stokes equations. 
The work studies the theoretical foundations of PINNs, conducts a 
comparative analysis with traditional numerical methods, and develops a specialized 
neural network architecture with adapted activation functions and optimization methods. 
The implemented model integrates boundary conditions and physical constraints, undergoes training 
and validation on test cases using analytical solutions. 
The scientific novelty of the approach lies in the direct integration of physical laws into the 
neural network architecture, ensuring compliance with fundamental physics principles. Practical 
advantages include reduced dependence on experimental data, high computational speed, 
and efficient use of computational resources (CPU, GPU, NPU). 
The method is capable of generating visually plausible solutions even with incomplete physical correspondence.

% Belarusian Translation
\chapter*{Рэферат}
\addcontentsline{toc}{chapter}{Рэферат}
Агульны аб'ём работы $40$ старонак, $21$ малюнак, $4$ табліцы, $33$ крыніцы.

Фізічна-інфармаваныя нейронныя сеткі, ўраўненні Наўе-Стокса, 
камп'ютэрнае мадэляванне, бясеткавы метад, цячэнне Куэтта.

У апошнія гады метады машыннага навучання і штучнага інтэлекту актыўна прымяняюцца 
для рашэння дыферэнцыяльных ураўненняў, якія ляжаць у аснове мадэлявання фізічных працэсаў. 
Асаблівую цікавасць уяўляюць фізічна-інфармаваныя нейронныя сеткі (PINN), якія 
аб'ядноўваюць магчымасці нейрасецяў з фундаментальнымі фізічнымі законамі. Мэтай даследавання 
была распрацоўка мадэлі руху вадкасці на аснове PINN для рашэння сістэмы ўраўненняў Наўе-Стокса. 
У працы вывучаны тэарэтычныя асновы PINN, праведзены іх 
параўнальны аналіз з традыцыйнымі лікавымі метадамі, распрацавана спецыялізаваная 
архітэктура нейрасеці з адаптаванымі функцыямі актывацыі і метадамі аптымізацыі. 
Рэалізаваная мадэль інтэгруе гранічныя ўмовы і фізічныя абмежаванні, прайшла навучанне 
і праверку на тэставых прыкладах з выкарыстаннем аналітычных рашэнняў. 
Навуковая навізна падыходу заключаецца ў прамой інтэграцыі фізічных законаў у архітэктуру 
нейрасеці, што забяспечвае выкананне фундаментальных прынцыпаў фізікі. Практычныя 
перавагі ўключаюць зніжэнне залежнасці ад эксперыментальных дадзеных, высокую хуткасць 
вылічэнняў і эфектыўнае выкарыстанне вылічальных рэсурсаў (CPU, GPU, NPU). 
Метад здольны генерыраваць візуальна даставерныя рашэнні нават пры няпоўным фізічным адпаведнасці.

\chapter*{Введение}
\addcontentsline{toc}{chapter}{Введение}

В последние годы методы машинного обучения и искусственного интеллекта находят всё
более широкое применение в различных областях науки и техники. Одним из перспективных
направлений является использование нейронных сетей для решения дифференциальных
уравнений и их систем, которые лежат в основе математического моделирования физических,
химических, биологических и других процессов. В этом контексте особый интерес представляют
физически-информированные нейронные сети (Physics-Informed Neural Networks, PINN) ---
гибридный подход, объединяющий мощь нейронных сетей с фундаментальными законами физики,
выраженными в виде дифференциальных уравнений.

Целью работы являлась разработка и исследование модели движения жидкости на основе
физически-информированных нейронных сетей для повышения точности и эффективности
численного моделирования гидродинамических процессов с учётом физических законов и
ограничений. В рамках исследования были изучены теоретические основы и математическая
формулировка физически-информированных нейронных сетей применительно к задачам
гидродинамики, проведён сравнительный анализ преимуществ и ограничений PINN по
сравнению с традиционными численными методами моделирования течений жидкости. Также
была разработана архитектура нейронной сети с использованием специализированных функций
активации и методов оптимизации, адаптированных для решения уравнений движения жидкости.
Реализована вычислительная модель, интегрирующая граничные условия и физические ограничения,
характерные для рассматриваемых гидродинамических систем. Проведены обучение и валидация
модели на тестовых примерах с использованием эталонных аналитических решений и
экспериментальных данных. В итоге определены перспективные направления развития
методологии физически-информированных нейронных сетей в контексте гидродинамического
моделирования.

Объектом исследования в данной работе является моделирование движения жидкости.
Предмет исследования — применение физически-информированных нейронных сетей
для повышения точности и эффективности численного моделирования гидродинамических
процессов с учётом физических законов и ограничений.

Научная новизна физически-информированных нейронных сетей заключается в интеграции
физических законов, описываемых дифференциальными уравнениями, непосредственно в
архитектуру нейронных сетей. Это позволяет моделям не только обучаться на данных,
но и строго соблюдать фундаментальные принципы физики, что значительно повышает
точность и устойчивость решений сложных задач гидродинамики и других областей.

Практическая значимость заключается в отсутствии необходимости в большом количестве
экспериментальных данных, в быстром воспроизведении решения, а также минимизации
объема информации, необходимом для воспроизведения решения с большим количеством
параметров. Также такой подход позволяет легко использовать параллельные вычисления
как на CPU, так и на GPU. С точки зрения гидродинамики это может быть полезно в игровой индустрии
для уменьшения нагрузки на устройство пользователя. Практически все новые мобильные
устройства содержат NPU, что позволяет на аппаратном уровне ускорить и облегчить
воспроизведение результатов обучения. Такой подход может найти применение и в 
киноиндустрии: «запекание» с возможностью подробной параметризации модели позволит
ускорить расчет для финальной сцены, а также ускоряют и упрощают процесс повторных съёмок.
В данном контексте следует отметить, что физически-информированные нейронные сети способны
генерировать визуально достоверные решения, несмотря на возможные отклонения от полного
физического соответствия.
