\chapter{Введение}

В последние годы методы машинного обучения и искусственного интеллекта находят всё
более широкое применение в различных областях науки и техники. Одним из перспективных
направлений является использование нейронных сетей для решения дифференциальных
уравнений и их систем, которые лежат в основе математического моделирования физических,
химических, биологических и других процессов. В этом контексте особый интерес представляют
Физически Информированные Нейронные Сети (Physics-Informed Neural Networks, PINN) —
гибридный подход, объединяющий мощь нейронных сетей с фундаментальными законами физики,
выраженными в виде дифференциальных уравнений.

PINN позволяют эффективно решать задачи, где традиционные численные методы сталкиваются
с трудностями, такими как высокая размерность данных, сложные граничные условия или
отсутствие достаточного количества экспериментальных данных. Этот подход основан на
включении физических законов непосредственно в процесс обучения нейронной сети, что
обеспечивает более точное и устойчивое решение задач моделирования.

Актуальность темы данной работы обусловлена растущим интересом к методам PINN в
научном сообществе, а также их потенциальной применимостью в гидродинамике для
решения задач, связанных с моделированием течений жидкости, турбулентности,
аэродинамики, океанографии и других сложных процессов. Цель работы заключается в
исследовании возможностей и ограничений PINN в контексте гидродинамики, разработке и
реализации алгоритмов для решения конкретных задач, а также в анализе их эффективности
по сравнению с традиционными методами.

В рамках работы будут рассмотрены теоретические основы PINN, проведены численные
эксперименты и предложены рекомендации по их применению в практических задачах.
Результаты исследования могут быть полезны для дальнейшего развития методов машинного
обучения в контексте математического моделирования и решения сложных научно-технических задач.


% \section{Проблема метода}
% Данный подход универсален для задач, решение которых гладкое и непрерывное во всей области решения. Для решений, содержащих
% разрывы, обычная модель будет выдавать значительную погрешность даже для большой модели с длительным обучением. Связано это с тем,
% что стандартные функции активации зачастую гладкие или непрерывные, что не позволяет нейронной сети нарушить гладкость
% результирующего решения. В таких случаях необходимо вводить новые активационные функции, либо использовать существующие,
% которые будут специфичны для конкретной задачи. Таким образом, для сложных задач следует использовать как минимум ReLU
% функцию активации в силу ее не гладкости(?). Далее в работе будет подробно описано влияение функции активации, количество слоев,
% нейронов и количество эпох на результат для наглядного изображения проблемы для случаев гладкого и негладкого(?) решения.

% % \section{Универсальная теорема об аппроксимации}
% % \section{Теорема Колмогорова-Арнольда}

% \section{Методы исследования}
% Че нибудь про технологии