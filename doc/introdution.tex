\chapter{Введение}

В последние годы методы машинного обучения и искусственного интеллекта находят всё
более широкое применение в различных областях науки и техники. Одним из перспективных
направлений является использование нейронных сетей для решения дифференциальных
уравнений и их систем, которые лежат в основе математического моделирования физических,
химических, биологических и других процессов. В этом контексте особый интерес представляют
физически-информированные нейронные сети (Physics-Informed Neural Networks, PINN) ---
гибридный подход, объединяющий мощь нейронных сетей с фундаментальными законами физики,
выраженными в виде дифференциальных уравнений.

\section{Цель и задачи исследования}
Цель данной работы разработать и исследовать модель движения жидкости на основе
физически-информированных нейронных сетей для повышения точности и эффективности
численного моделирования гидродинамических процессов с учётом физических законов
и ограничений.

Также были поставлены следующие задачи:
\begin{enumerate}
    \item Изучить теоретические основы и математическую формулировку физически-информированных
    нейронных сетей применительно к задачам гидродинамики.
    Проанализировать преимущества и ограничения PINN по сравнению с традиционными численными
    методами моделирования течений жидкости.
    
    \item Разработать архитектуру нейронной сети с использованием подходящих функций активации и
    методов оптимизации для решения уравнений движения жидкости.
    
    \item Реализовать вычислительную модель с учётом граничных условий, а также
    физических ограничений, характерных для рассматриваемых гидродинамических задач.
    
    \item Провести обучение и валидацию модели на тестовых примерах с известными аналитическими
    или экспериментальными решениями.
    
    \item Выявить перспективы и направления дальнейшего развития применения  в
    гидродинамическом моделировании.
\end{enumerate}
\section{Объект и предмет исследования}
Объектом исследования в данной работе является моделирование движения жидкости.

Предмет исследования — применение физически-информированных нейронных сетей
для повышения точности и эффективности численного моделирования гидродинамических
процессов с учётом физических законов и ограничений.
\section{Методы исследования}
В данном исследовании для моделирования движения жидкости с помощью физически
информированных нейронных сетей использованы следующие методы:
\begin{enumerate}
    \item Анализ основных уравнений гидродинамики и их математическая формулировка
    с учётом физических законов.
    \item Разработка архитектуры нейронной сети с подбором функций активации и
    методов оптимизации, подходящих для решения уравнений движения жидкости.
    \item Реализация вычислительной модели с учётом граничных условий и физических
    ограничений, характерных для гидродинамических задач.
    \item Обучение и проверка модели на тестовых данных с известными аналитическими
    решениями.
    \item Изучение влияния параметров модели, таких как структура сети, методы
    оптимизации, количество точек внутри области и скорость обучения, на качество
    и эффективность решения.
    \item Данный подход позволяет создавать точные и устойчивые модели для решения
    задач гидродинамики с использованием нейронных сетей.
\end{enumerate}

\section{Научная новизна и практическая значимость}
Научная новизна физически информированных нейронных сетей заключается в интеграции
физических законов, описываемых дифференциальными уравнениями, непосредственно в
архитектуру нейронных сетей. Это позволяет моделям не только обучаться на данных,
но и строго соблюдать фундаментальные принципы физики, что значительно повышает
точность и устойчивость решений сложных задач гидродинамики и других областей.

Практическая значимость заключается в отсутствии необходимости в большом количестве
эксперементальных данных, в быстром воспроизведении решения, а также минимизации
объема информации, необходимом для воспроизведения решения с большим количеством
параметров. Также такой подход позволяет легко использовать параллельные вычисления
как на CPU, так и на GPU. С точки зрения гидродинамики это может быть полезно в игровой индустрии
для уменьшения нагрузки на устройство пользователя. Практически все новые мобильные
устройства содержат NPU, что позволяет на аппаратном уровне ускорить и облегчить
воспроизведение результатов обучения. Такой подход может найти применение и в 
киноиндустрии: «запекание» с возможностью подробной параметризации модели позволит
ускорить расчет для финальной сцены, а также ускоряют и упрощают процесс повторных съёмок.
В данном контексте следует отметить, что физически информированные нейронные сети способны
генерировать визуально достоверные решения, несмотря на возможные отклонения от полного
физического соответствия.
