\chapter{Введение}

В последние годы методы машинного обучения и искусственного интеллекта находят всё
более широкое применение в различных областях науки и техники. Одним из перспективных
направлений является использование нейронных сетей для решения дифференциальных
уравнений и их систем, которые лежат в основе математического моделирования физических,
химических, биологических и других процессов. В этом контексте особый интерес представляют
Физически Информированные Нейронные Сети (Physics-Informed Neural Networks, PINN) ---
гибридный подход, объединяющий мощь нейронных сетей с фундаментальными законами физики,
выраженными в виде дифференциальных уравнений.

\section{Цель и задачи исследования}
Цель данной работы разработать и исследовать модель движения жидкости на основе
физически-информированных нейронных сетей для повышения точности и эффективности
численного моделирования гидродинамических процессов с учётом физических законов
и ограничений.

Также были поставлены следующие задачи:
\begin{enumerate}
    \item Изучить теоретические основы и математическую формулировку физически-информированных
    нейронных сетей применительно к задачам гидродинамики.
    Проанализировать преимущества и ограничения PINN по сравнению с традиционными численными
    методами моделирования течений жидкости.
    
    \item Разработать архитектуру нейронной сети с использованием подходящих функций активации и
    методов оптимизации для решения уравнений движения жидкости.
    
    \item Реализовать вычислительную модель с учётом граничных условий, а также
    физических ограничений, характерных для рассматриваемых гидродинамических задач.
    
    \item Провести обучение и валидацию модели на тестовых примерах с известными аналитическими
    или экспериментальными решениями.
    
    \item Выполнить моделирование сложных гидродинамических сценариев (например, турбулентных
    течений, многофазных потоков) и сравнить результаты с традиционными методами.
    
    \item Выявить перспективы и направления дальнейшего развития применения  в
    гидродинамическом моделировании.
\end{enumerate}
\section{Объект и предмет исследования}
В данной работе исследуется двумерное течение жидкости в каналах различной геометрии,
описываемое уравнениями Навье–Стокса.

Процессы моделирования и численного решения уравнений Навье–Стокса для двумерных течений
жидкости в каналах с использованием физически-информированных нейронных сетей, включая
разработку, обучение и валидацию соответствующих моделей.

\section{Методы исследования}

\section{Научная новизна и практическая значимость}



% PINN позволяют эффективно решать задачи, где традиционные численные методы сталкиваются
% с трудностями, такими как высокая размерность данных, сложные граничные условия или
% отсутствие достаточного количества экспериментальных данных. Этот подход основан на
% включении физических законов непосредственно в процесс обучения нейронной сети, что
% обеспечивает более точное и устойчивое решение задач моделирования.

% Актуальность темы данной работы обусловлена растущим интересом к методам PINN в
% научном сообществе, а также их потенциальной применимостью в гидродинамике для
% решения задач, связанных с моделированием течений жидкости, турбулентности,
% аэродинамики, океанографии и других сложных процессов. Цель работы заключается в
% исследовании возможностей и ограничений PINN в контексте гидродинамики, разработке и
% реализации алгоритмов для решения конкретных задач, а также в анализе их эффективности
% по сравнению с традиционными методами.

% В рамках работы будут рассмотрены теоретические основы PINN, проведены численные
% эксперименты и предложены рекомендации по их применению в практических задачах.
% Результаты исследования могут быть полезны для дальнейшего развития методов машинного
% обучения в контексте математического моделирования и решения сложных научно-технических задач.


% \section{Проблема метода}
% Данный подход универсален для задач, решение которых гладкое и непрерывное во всей области решения. Для решений, содержащих
% разрывы, обычная модель будет выдавать значительную погрешность даже для большой модели с длительным обучением. Связано это с тем,
% что стандартные функции активации зачастую гладкие или непрерывные, что не позволяет нейронной сети нарушить гладкость
% результирующего решения. В таких случаях необходимо вводить новые активационные функции, либо использовать существующие,
% которые будут специфичны для конкретной задачи. Таким образом, для сложных задач следует использовать как минимум ReLU
% функцию активации в силу ее не гладкости(?). Далее в работе будет подробно описано влияение функции активации, количество слоев,
% нейронов и количество эпох на результат для наглядного изображения проблемы для случаев гладкого и негладкого(?) решения.

% % \section{Универсальная теорема об аппроксимации}
% % \section{Теорема Колмогорова-Арнольда}

% \section{Методы исследования}
% Че нибудь про технологии