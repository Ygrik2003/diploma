% \chapter{Метод исследования}
% Нейронные сети весьма непредсказуемы при обучении, особенно в случае отсутствия
% обучающих и тестовых данных. Для понимания работы нейронных сетей в таком режиме
% следует изучить влияние количества нейронов, активационной функции, количества слоев, а также
% влияние Dropout'ов на такую нейронную сеть.

% Для исследования поведения модели в решении дифференциальных уравнений в частных
% производных будем использовать две различных системы для качественной оценки.
% Первая модель будет представлять собой решение двумерной задачи для уравнений Навье-Стокса.
% Вторая модель --- трехмерной задачи для уравнений Навье-Стокса.
% % Нужно добавить еще задачи с обтеканием препятствия и явно указать тип жидкости.

% В качестве задачи для первой модели будем использовать следующую задачу:
% Для прямоугольной области запишем уравнения Навье-Стокса для несжимаемой жидкости
% \begin{equation}
% 	\begin{cases}
% 		\frac{\partial{u}}{\partial{t}} = D \cdot \left(\frac{\partial^2{u}}{\partial{x^2}} +
% 		\frac{\partial^2{u}}{\partial{y^2}}\right) \\
% 		\frac{\partial{u_x}}{\partial{x}} + \frac{\partial{u_y}}{\partial{y}} = 0
% 	\end{cases}
% \end{equation}

% % Явно прописать задачи с формулами

% Такой подход позволит продемонстрировать возрастание сложности модели, а также сравнить возрастание
% сложности на фоне численного метода и оценить погрешность безсеточного метода.
