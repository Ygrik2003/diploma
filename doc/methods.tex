\chapter{Методы}
С точки зрения машинного обучения и нейронных сетей есть множество важных аспектов,
помимо точности. В контексте физически-информированных нейронных сетей речь пойдет
об оптимизации архитектуры нейронной сети для ускорения обучения, увеличения точности
для оценки компонент скоростей или давления \cite{Tommaso2024pinn}.

% MFN-PINN,  MLP-PINN
В первую очередь нужно обратить внимание на функции активации, которые определяют,
как нейрон будет реагировать на входные данные, и могут существенно влиять на способность
сети обучаться сложным функциям и обобщать полученные знания.

Каждый нейрон в нейронной сети использует функцию активации для преобразования взвешенной
суммы своих входных сигналов в выходной сигнал. Эта функция вносит нелинейность в модель,
что необходимо для обучения сложных зависимостей.

Формула, описывающая этот процесс, выглядит следующим образом:

$$y = f(\sum_{i=1}^{n} w_i x_i + b)$$

где $x_i$ — $i$-й входной сигнал нейрона, $w_i$ — вес, связанный с $i$-м входным сигналом, $b$
— смещение нейрона, $f(\cdot)$ — функция активации, $y$ — выходной сигнал нейрона.

Для эффективного использования в PINNs функции активации должны удовлетворять следующим
критериям \cite{0d752c79fb816703274a3d37f85a85689a2a9405}:
\begin{itemize}
    \item функции активации должны быть гладкими и
    непрерывно дифференцируемыми, чтобы обрабатывать функции потерь, которые включают
    производные высших порядков.
    \item функции активации должны допускать неограниченные
    выходные значения, в отличие от функций $tanh$ и $sin$, которые ограничены между $-1$ и $1$.
    \item функции активации должны избегать насыщения, чтобы предотвратить
    исчезновение градиентов, что может затруднить обучение.
    \item в некоторых случаях желательно иметь контролируемое насыщение за пределами определённого
    диапазона, чтобы улучшить способность модели представлять сложные физические сигналы.
\end{itemize}


\subsection{ReLU}
\begin{figure}[ht]
    \centering
    \begin{tikzpicture}
        \begin{axis}[
            width=14cm, height=10cm,
            grid=both,
            axis y line=middle, axis x line=middle,
            ymin=-1.2, ymax=1.2,
            xmin=-5, xmax=5,
            samples=201,
            legend style={at={(0.5,-0.1)},anchor=north,legend columns=2},
            legend cell align={left},
            title={Активационные функции}
        ]
        
        % Пороговая функция (Heaviside, Step)
        \addplot[thick, blue, domain=-5:5] {x >= 0 ? 1 : 0};
        \addlegendentry{Пороговая (Heaviside)}
        
        % Сигмоида (Logistic Sigmoid)
        \addplot[thick, red, domain=-5:5] {1/(1+exp(-x))};
        \addlegendentry{Сигмоида (Logistic)}
        
        % Гиперболический тангенс (tanh)
        \addplot[thick, green!70!black, domain=-5:5] {tanh(x)};
        \addlegendentry{Гиперб. тангенс (tanh)}
        
        % ReLU
        \addplot[thick, orange, domain=-5:5] {x > 0 ? x : 0};
        \addlegendentry{ReLU}
        
        % Сигмоида сдвинутая (Bipolar Sigmoid)
        \addplot[thick, magenta, domain=-5:5] {(1 - exp(-x))/(1 + exp(-x))};
        \addlegendentry{Сигмоида сдвинутая (Bipolar)}
        
        % Гауссова функция
        \addplot[thick, cyan, domain=-5:5] {exp(-x^2)};
        \addlegendentry{Гауссова функция}
        
        \end{axis}
    \end{tikzpicture}
    \caption{Функции активации на основе Swish}
    \label{fig:act_func_graph}    
\end{figure}
\begin{figure}[ht]
    \caption{Структура PINN с использованием функции активации Adaptive Blending Unit\cite{a104fe01d341f235fd80ea98d6a8f35b8110df1d}}
    \label{pinn_structure}
    \centering
    \resizebox{0.8\columnwidth}{!}{
        \begin{tikzpicture}[node distance=1.2cm, thick]

            % Цвета нод
            \definecolor{inputcolor}{RGB}{255,230,230}   % светло-розовый
            \definecolor{hiddencolor}{RGB}{200,220,255}  % светло-голубой
            \definecolor{outputcolor}{RGB}{220,255,220}  % светло-зелёный
            
            % Входной слой
            \foreach \i in {1,2,3}
                \node[draw, circle, fill=inputcolor, minimum size=1cm] (I\i) at (0,1.8-\i*1.2) {};
            
            % Скрытый слой 1
            \foreach \i in {1,2,3,4}
                \node[draw, circle, fill=hiddencolor, minimum size=1cm] (Hone\i) at (2.5,2.4-\i*1.2) {};
            
            % Скрытый слой 2
            \foreach \i in {1,2,3,4}
                \node[draw, circle, fill=hiddencolor, minimum size=1cm] (Htwo\i) at (5.5,2.4-\i*1.2) {};
            
            % Выходной слой
            \node[draw, circle, fill=outputcolor, minimum size=1cm] (O) at (8.2,-0.6) {};
            
            % Связи: входной -> скрытый 1
            \foreach \i in {1,2,3}
                \foreach \j in {1,2,3,4}
                    \draw[->] (I\i) -- (Hone\j);
            
            % Связи: скрытый 1 -> скрытый 2
            \foreach \i in {1,2,3,4}
                \foreach \j in {1,2,3,4}
                    \draw[->] (Hone\i) -- (Htwo\j);
            
            % Связи: скрытый 2 -> выходной
            \foreach \i in {1,2,3,4}
                \draw[->] (Htwo\i) -- (O);
            
            % Подписи слоёв
            \node[align=center] at (-0.2,2.8) {Входной\\слой};
            \node[align=center] at (2.5,2.8) {Скрытый\\слой 1};
            \node[align=center] at (5.5,2.8) {Скрытый\\слой 2};
            \node[align=center] at (8.2,2.8) {Выходной\\слой};
        \end{tikzpicture}       
    }
\end{figure}
ReLU (Rectified Linear Unit) — это функция активации, которая возвращает входное значение,
если оно положительное, и 0, если оно отрицательное \eqref{relu}. Это одна из самых популярных 
функций активации в нейронных сетях, благодаря своей простоте и эффективности. 
\begin{equation}
    \text{ReLU}(x) = \max(0, x)
    \label{relu}
\end{equation}
В контексте физически информированных нейронных сетей данная функция не эффективна в силу
разрава при $x = 0$ и производной, равной $1$ при положительных значениях $x$, что создает
проблемы при обратном распространении ошибки.

\subsection{SiLU}
SiLU (Sigmoid Linear Unit), также известная как $\text{Swish-}1$ \eqref{silu}, является
гладкой немонотонной функцией активации, используемой в нейронных сетях.
\begin{equation}
    \text{SiLU}(x) = x \cdot \sigma(x) \label{silu}
\end{equation}
Эта функция удобна своей гладкостью и способностью улучшать производительность глубоких
моделей машинного обучения по сравнению с функциями активации, такими как ReLU.
\subsection{Swish}
Swish является гладкой и немонотонной функцией, аналогично SiLU, но имеет дополнительный
параметр $\beta$, который может быть настроен или зафиксирован \eqref{swish}.
\begin{equation}
    \label{swish}
    swish_\beta(x) = \frac{x}{1 + e^{-\beta x}}
\end{equation}
В целом Swish является заменой всех предыдущих функций:
\begin{itemize}
    \item При $\text{swish}_1(x) = \text{SiLU}(x)$
    \item При $\text{swish}\inf(x) = \text{ReLU}(x)$
\end{itemize}

GELU (Gaussian Error Linear Unit) и
SILU (Sigmoid Linear Unit), а также стандартные ReLU (Rectified Linear Unit), Sigmoid
и Tanh. Помимо обычных функций активаций есть адаптивные, которые подразумевают подбор
специфичных коэфициентов в процессе обучения нейронной сети для улучшения "понимания"
физической составляющей искомой задачи. К таким функциям относятся функции ABU
(Adaptive Blending Unit) и REAct (Rational Exponential Activation)

