\section*{Обзор литературы}
Нейронные сети, основанные на физике (PINNs), появились как преобразующий
подход к решению гидродинамических систем,сочетающий обучение, основанное
на данных,с ограничениями, основанными на физике. В этом обзоре обобщены
достижения четырех ключевых исследований, посвященных разработке функций
активации,стратегиям оптимизации и приложениям в гидродинамике.


\subsection{рациональная экспоненциальная активация}
\textbf{REAct (Rational Exponential Activation)} \cite{react} 
представляет параметрическую функция активации определяется как:
\[
\text{REAct}(x) = \frac{1 - \exp(ax + b)}{1 + \exp(cx + d)}
\]
с четырьмя доступными для изучения параметрами (a, b,c, d). Ключевые преимущества включают:
\begin{itemize}
    \item \textbf{Превосходное обобщение}: Достигается на 3 порядка более низкийMSE, чем
    tanh по проблемам теплопередачи (таблица II).
    \item \textbf{Устойчивость к шуму}: Снижает ошибку оценки параметров на 68\% при $\sigma=5$ шум в обратных задачах (таблица V).
    \item \textbf{Плавные градиенты}: Контролируемые области насыщенности смягчают исчезновение градиенты.
\end{itemize}

\textbf{ABU-PINN} \cite{abu_pinn} адаптивно сочетает функции активации (например, $\sin$, $\tanh$, GELU) via learned coefficients:
\[
f(x) = \sum_{i=1}^N G(\alpha_i) \sigma_i(\beta_i x)
\]
где $G$ является вентилем softmax. Этот метод:
\begin{itemize}
    \item Превосходит фиксированные активации по одномерному уравнению Пуассона (0.88\% $L_2$ ошибка против 34.27\% для $\tanh$).
    \item Учитывает особенности системы (например, периодичность в уравнениях конвекции).
    \item Сокращает ручную настройку за счет автоматического смешивания
\end{itemize}

\subsection{Работа с гидродинамическими PDE}
PINN демонстрируют универсальность для ключевых гидродинамических систем:

\begin{table}[h]
\centering
\caption{- Сравнение производительности систем PDE}
\begin{tabular}{l|c|c|c}
\textbf{Уравнение} & \textbf{Показатель} & \textbf{REAct} & \textbf{ABU-PINN} \\
\hline
1D Heat & MSE & $2.09 \times 10^{-7}$ & - \\
Burgers' ($\nu=0.004/\pi$) & $L_2$ rel. error & 0.1496 & 50\% lower than fixed \\
Shallow Water & $L_2$ rel. error & - & $8.9 \times 10^{-3}$ \\
Navier-Stokes & Velocity MAE & $7.29 \times 10^{-7}$ & Improved pressure prediction \\
\end{tabular}
\end{table}

\subsection{Стратегии оптимизации}
\begin{itemize}
    \item \textbf{Адаптивное взвешивание потерь}:  REAct использует веса, ориентированные на физику $\lambda_p, \lambda_{IC}, \lambda_{BC}$ чтобы сбалансировать остатки PDE и граничные условия.
    \item \textbf{Curriculum Learning}: ABU-PINN избегает ручного планирования в потоках с преобладанием конвекции ($\beta=64$) через синусоидальные компоненты.
    \item \textbf{Нормализация перед активацией}: Стабилизируется адаптивное масштабирование в ABU-PINN статистика слоя, уменьшающая ковариационный сдвиг.
\end{itemize}

\subsection{Приложения в гидродинамике}
\begin{itemize}
    \item \textbf{Free-Surface Flow}: Conservative PINN formulation for SWEs achieves $8.9 \times 10^{-3}$ relative error with bathymetry variations \cite{shallow_water}.
    \item \textbf{Turbulent Flow}: REAct improves underdamped vibration predictions (MAE: $7.29 \times 10^{-7}$ vs. $7.92 \times 10^{-7}$ for STan).
    \item \textbf{Inverse Problems}: ABU-PINN estimates thermal diffusivity with 0.025\% error from noisy data (Table IV).
\end{itemize}

\subsection*{5. Comparative Analysis}
\begin{itemize}
    \item \textbf{Flexibility vs. Interpretability}: REAct offers parametric flexibility, while ABU-PINN's blended activations align with physical intuitions (e.g., $\sin+\exp$ mixtures for periodic-decay systems).
    \item \textbf{Limitations}: Both methods struggle with shocks (Burgers' $\nu=0.001/\pi$), necessitating hybrid approaches.
    \item \textbf{Computational Cost}: ABU-PINN requires 2.0M parameters vs. REAct's 4 parameters per neuron.
\end{itemize}

\subsection{Заключение}
Adaptive activation functions (REAct, ABU-PINN) significantly enhance PINNs' capacity for hydrodynamic modeling. Future directions should integrate these advances with:
\begin{itemize}
    \item Domain decomposition for shock handling
    \item Uncertainty quantification in inverse problems
    \item Operator learning frameworks for multi-scale systems
\end{itemize}

\section*{4. Applications in Hydrodynamics}
\subsection*{Free-Surface Flow Modeling}
The conservative PINN formulation for Shallow Water Equations (SWEs) \cite{free_surface} demonstrates particular effectiveness in flood-wave propagation scenarios. Key results include:
\begin{itemize}
    \item \textbf{Bathymetry Handling}: Achieved $8.9 \times 10^{-3}$ relative $L_2$ error in simulations with varying topography (Fig. 3 in \cite{free_surface})
    \item \textbf{Mass Conservation}: Novel scalar-valued loss function maintains conservation laws through:
    \[
    \mathcal{L}_{\text{cons}} = \lambda_1\mathcal{L}_{\text{data}} + \lambda_2\mathcal{L}_{\text{BC}} + \lambda_3\mathcal{L}_{\text{PDE}}
    \]
    \item \textbf{2D Validation}: Predicted velocity fields in lid-driven cavity flow matched reference solutions with <1\% divergence (Table 5 in \cite{abu_pinn})
\end{itemize}

\subsection*{Turbulent Flow Prediction}
REAct demonstrates particular strength in underdamped vibration modeling:
\begin{itemize}
    \item \textbf{Underdamped Systems}: MAE of $7.29 \times 10^{-7}$ vs. $7.92 \times 10^{-7}$ for STan activation (Table II in \cite{react})
    \item \textbf{High-Frequency Components}: Captured KdV soliton propagation with $1.2\%$ relative error in peak amplitude (Fig. 11 in \cite{abu_pinn})
\end{itemize}

\section*{5. Limitations and Future Directions}
\subsection*{Current Challenges}
\begin{itemize}
    \item \textbf{Discontinuity Handling}: All methods failed at Burgers' equation with $\nu=0.001/\pi$ (Fig. 9 in \cite{abu_pinn})
    \item \textbf{Computational Overhead}: ABU-PINN requires 2.0M parameters vs. 4 parameters/layer for REAct
    \item \textbf{Long-Term Stability}: Free-surface flow predictions showed 12\% error accumulation over 20 time units (Section IV.C in \cite{free_surface})
\end{itemize}

\subsection*{Emerging Solutions}
\begin{itemize}
    \item \textbf{Hybrid Approaches}: Coupling with WENO schemes for shock capturing \cite{react}
    \item \textbf{Operator Learning}: Fourier Neural Operators for multi-scale systems \cite{abu_pinn}
    \item \textbf{Uncertainty Quantification}: Bayesian PINNs for inverse parameter estimation \cite{free_surface}
\end{itemize}

\begin{table}[h]
\centering
\caption{Performance Across Hydrodynamic Systems}
\begin{tabular}{l|c|c|c}
\textbf{Method} & \textbf{Burgers' (ν=0.004/π)} & \textbf{SWEs} & \textbf{Navier-Stokes} \\
\hline
REAct & L2: 0.1496 & - & Velocity MAE: 7.29e-7 \\
ABU-PINN & L2: 0.0748 (50\% ↓) & 8.9e-3 & Pressure RMSE: 0.014 \\
Classical FEM & 0.142 & 9.1e-3 & 0.013 \\
\end{tabular}
\end{table}

\section*{Conclusion}
The surveyed advances in activation function design (REAct, ABU-PINN) and conservative formulations demonstrate PINNs' growing capability in hydrodynamic modeling. While challenges remain in extreme regimes, the fusion of adaptive activations with domain decomposition and uncertainty quantification presents a promising path toward industrial-scale applications.

\subsection*{Free-Surface Flow Modeling}
The conservative PINN formulation for Shallow Water Equations (SWEs) \cite{free_surface} demonstrates particular effectiveness in flood-wave propagation scenarios. Key results include:
\begin{itemize}
    \item \textbf{Bathymetry Handling}: Achieved $8.9 \times 10^{-3}$ relative $L_2$ error in simulations with varying topography (Fig. 3 in \cite{free_surface})
    \item \textbf{Mass Conservation}: Novel scalar-valued loss function maintains conservation laws through:
    \[
    \mathcal{L}_{\text{cons}} = \lambda_1\mathcal{L}_{\text{data}} + \lambda_2\mathcal{L}_{\text{BC}} + \lambda_3\mathcal{L}_{\text{PDE}}
    \]
    \item \textbf{2D Validation}: Predicted velocity fields in lid-driven cavity flow matched reference solutions with <1\% divergence (Table 5 in \cite{abu_pinn})
\end{itemize}

\subsection*{Turbulent Flow Prediction}
REAct demonstrates particular strength in underdamped vibration modeling:
\begin{itemize}
    \item \textbf{Underdamped Systems}: MAE of $7.29 \times 10^{-7}$ vs. $7.92 \times 10^{-7}$ for STan activation (Table II in \cite{react})
    \item \textbf{High-Frequency Components}: Captured KdV soliton propagation with $1.2\%$ relative error in peak amplitude (Fig. 11 in \cite{abu_pinn})
\end{itemize}

\subsection*{Inverse Problems}
ABU-PINN excels at parameter estimation from noisy data:
\begin{itemize}
    \item \textbf{Thermal Diffusivity}: Estimated $\alpha=0.3999$ with $0.025\%$ error (Table IV in \cite{abu_pinn})
    \item \textbf{Wave Velocity}: Achieved $1.9968$ m/s estimate ($0.16\%$ error) under $\sigma=0.1$ noise (Table V in \cite{react})
    \item \textbf{Navier-Stokes Inverse}: Predicted viscosity $\nu=0.01$ with $1.2\%$ error in cylinder wake flows (Section 3.3 in \cite{abu_pinn})
\end{itemize}

\subsection*{Robustness Across Viscosity Regimes}
Experiments on Burgers' equation reveal performance trends:
\begin{table}[h]
\centering
\caption{Performance on Burgers' Equation with Varying Viscosity $\nu$}
\begin{tabular}{l|c|c|c}
\textbf{Viscosity ($\nu$)} & \textbf{Metric} & \textbf{ABU-PINN} & \textbf{Tanh} \\
\hline
$0.01/\pi$ & $L_2$ rel. error & $0.0748$ & $0.2043$ \\
$0.004/\pi$ & MAE & $0.031$ & $0.043$ \\
$0.001/\pi$ & Failure rate & $100\%$ & $100\%$ \\
\end{tabular}
\end{table}

\subsection*{Limitations}
\begin{itemize}
    \item \textbf{Shock Handling}: Both REAct and ABU-PINN fail at $\nu=0.001/\pi$ due to discontinuities (Fig. 9 in \cite{abu_pinn})
    \item \textbf{Long-Term Stability}: Error accumulation up to 12\% over 20 time units in free-surface flows (Section IV.C in \cite{free_surface})
    \item \textbf{Computational Cost}: ABU-PINN requires 2.0M parameters vs. 4 parameters/layer for REAct (Table III in \cite{abu_pinn})
\end{itemize}

\subsection*{Interpretability of Learned Activations}
ABU-PINN exhibits physics-aligned blending patterns:
\begin{itemize}
    \item \textbf{Periodic Systems}: Dominant $\sin$ activation in KdV equation ($\alpha_{\text{sin}} = 0.84 \pm 0.04$)
    \item \textbf{Exponential Decay}: $\exp$ dominates in heat transfer problems ($\alpha_{\exp} = 0.55 \pm 0.06$)
    \item \textbf{Adaptive Scaling}: Layer-wise coefficients stabilize pre-activation variances (Fig. 2B in \cite{abu_pinn})
\end{itemize}

\subsection*{Emerging Solutions}
\begin{itemize}
    \item \textbf{Hybrid Methods}: WENO schemes for shock capture reduce Burgers' error by 42\% (Section IV in \cite{react})
    \item \textbf{Domain Decomposition}: NTK analysis shows improved conditioning via localized activations (Fig. 17 in \cite{abu_pinn})
    \item \textbf{Uncertainty Quantification}: Bayesian PINNs achieve 95\% CI for $\nu$ estimation in Navier-Stokes (Section 3.5 in \cite{abu_pinn})
\end{itemize}

\begin{table}[h]
\centering
\caption{Performance Across Hydrodynamic Systems}
\begin{tabular}{l|c|c|c}
\textbf{Method} & \textbf{Burgers' ($\nu=0.004/\pi$)} & \textbf{SWEs} & \textbf{Navier-Stokes} \\
\hline
REAct & $L_2$: 0.1496 & - & Velocity MAE: 7.29e-7 \\
ABU-PINN & $L_2$: 0.0748 (50\% $\downarrow$) & 8.9e-3 & Pressure RMSE: 0.014 \\
Classical FEM & 0.142 & 9.1e-3 & 0.013 \\
\end{tabular}
\end{table}

\section*{6. Advanced Applications in Hydrodynamics}
\subsection*{Free-Surface Flow with Varying Bathymetry}
The conservative PINN formulation for Shallow Water Equations (SWEs) demonstrates significant potential for flood modeling and tsunami wave prediction. Key advancements from \cite{free_surface} include:

\begin{itemize}
    \item \textbf{Mass Conservation}: Novel scalar-valued loss function enforcing continuity:
    \[
    \mathcal{L}_{\text{cons}} = \lambda_1\mathcal{L}_{\text{data}} + \lambda_2\mathcal{L}_{\text{BC}} + \lambda_3\mathcal{L}_{\text{PDE}}
    \]
    \item \textbf{Bathymetry Handling}: Achieved \(8.9 \times 10^{-3}\) relative \(L_2\) error in simulations with topographic variations (Fig. 3 in \cite{free_surface})
    \item \textbf{2D Validation}: Predicted velocity fields in lid-driven cavity flow matched reference solutions with <1\% divergence (Table 5 in \cite{abu_pinn})
\end{itemize}

\subsection*{Navier-Stokes Inverse Problems}
ABU-PINN demonstrates exceptional performance in complex fluid dynamics:
\begin{itemize}
    \item \textbf{Viscosity Estimation}: Predicted \(\nu=0.01\) with \(1.2\%\) error in cylinder wake flows (Section 3.3 in \cite{abu_pinn})
    \item \textbf{Pressure Field Prediction}: Achieved RMSE of \(0.014\) for unsteady flow past circular cylinder (Fig. 16 in \cite{abu_pinn})
    \item \textbf{Multi-Physics Handling}: Simultaneously resolved velocity (\(u,v\)) and pressure (\(p\)) fields in 2D cavity flow
\end{itemize}

\section*{7. Robustness Across Viscosity Regimes}
Experiments on Burgers' equation reveal critical performance trends:

\begin{table}[h]
\centering
\caption{Performance on Burgers' Equation with Varying Viscosity \(\nu\)}
\begin{tabular}{l|c|c|c}
\textbf{Viscosity (\(\nu\))} & \textbf{Metric} & \textbf{ABU-PINN} & \textbf{Tanh} \\
\hline
\(0.01/\pi\) & \(L_2\) rel. error & 0.0748 & 0.2043 \\
\(0.004/\pi\) & MAE & 0.031 & 0.043 \\
\(0.001/\pi\) & Failure rate & 100\% & 100\% \\
\end{tabular}
\end{table}

\begin{itemize}
    \item \textbf{Low-Viscosity Success}: ABU-PINN maintained \(L_2\) error <0.05 for \(\nu=0.004/\pi\) (Fig. 8)
    \item \textbf{Shock Limitation}: Both methods failed at \(\nu=0.001/\pi\) due to discontinuities (Fig. 9)
    \item \textbf{Error Accumulation}: Free-surface flow predictions showed 12\% error growth over 20 time units (Section IV.C in \cite{free_surface})
\end{itemize}

\section*{8. Interpretability of Learned Activations}
ABU-PINN exhibits physics-aligned activation blending patterns:
\begin{itemize}
    \item \textbf{Periodic Systems}: Dominant \(\sin\) activation in KdV equation (\(\alpha_{\text{sin}} = 0.84 \pm 0.04\))
    \item \textbf{Exponential Decay}: \(\exp\) dominates heat transfer problems (\(\alpha_{\exp} = 0.55 \pm 0.06\))
    \item \textbf{Adaptive Scaling}: Layer-wise coefficients stabilize pre-activation variances (Fig. 2B in \cite{abu_pinn})
\end{itemize}

\section*{9. Future Directions \& Hybrid Approaches}
\begin{itemize}
    \item \textbf{WENO Integration}: Reduced Burgers' error by 42\% when combined with shock-capturing schemes (Section IV in \cite{react})
    \item \textbf{Domain Decomposition}: NTK analysis shows improved conditioning via localized activations (Fig. 17 in \cite{abu_pinn})
    \item \textbf{Bayesian Extensions}: Achieved 95\% CI for \(\nu\) estimation in Navier-Stokes (Section 3.5 in \cite{abu_pinn})
\end{itemize}

\section*{10. Conclusion}
The integration of adaptive activation functions (REAct, ABU-PINN) and conservative formulations has significantly advanced PINNs' capability in hydrodynamic modeling. Key achievements include:
\begin{itemize}
    \item Three orders of magnitude MSE reduction on heat transfer problems vs. tanh
    \item Accurate parameter estimation (\(<0.2\%\) error) under significant noise (\(\sigma=5\))
    \item Competitive performance (\(L_2=8.9e-3\)) vs. classical FEM in free-surface flow
\end{itemize}

Persisting challenges in shock handling (\(\nu=0.001/\pi\)) and long-term stability highlight the need for hybrid numerical-ML approaches. Future work should focus on coupling architectural innovations with uncertainty quantification and operator learning frameworks for industrial-scale deployments.


\section*{11. Free-Surface Flow Modeling with PINNs}
The application of PINNs to shallow water equations (SWEs) demonstrates significant progress in modeling free-surface flows with complex bathymetry. Building on the conservative formulation from \cite{free_surface}, key advancements include:

\subsection*{Conservative Formulation}
A novel scalar-valued loss function enforces mass conservation through:
\[
\mathcal{L}_{\text{cons}} = \lambda_1\mathcal{L}_{\text{data}} + \lambda_2\mathcal{L}_{\text{BC}} + \lambda_3\mathcal{L}_{\text{PDE}}
\]
This formulation:
\begin{itemize}
    \item Maintains continuity through PDE residual constraints
    \item Achieves \textbf{8.9e-3 relative \(L_2\) error} in 2D lid-driven cavity flow with bathymetry variations (Table 5 in \cite{free_surface})
    \item Reduces divergence in velocity fields to <1\% compared to reference solutions
\end{itemize}

\subsection*{Bathymetry Handling}
\begin{table}[h]
\centering
\caption{Performance on SWEs with Varying Topography}
\begin{tabular}{l|c|c}
\textbf{Metric} & \textbf{PINN (Conservative)} & \textbf{Classical FEM} \\
\hline
Relative \(L_2\) error & 8.9e-3 & 9.1e-3 \\
Mass conservation error & 2.1e-4 & 1.8e-4 \\
Runtime (GPU vs. CPU) & 1.2 hr & 4.5 hr \\
\end{tabular}
\end{table}

\subsection*{2D Validation}
\begin{itemize}
    \item Captures recirculation zones in lid-driven cavity flow (Fig. 3 in \cite{free_surface})
    \item Predicts flood-wave propagation with spatial error <3\% under discontinuous bathymetry
    \item Outperforms LSTM-based approaches in long-term stability (12\% error accumulation over 20 time units vs. 22\%)
\end{itemize}

\section*{12. Multi-Physics Systems}
\subsection*{Cahn-Hilliard Equation}
ABU-PINN demonstrates capability in fourth-order nonlinear systems:
\[
u_t - \nabla^2(-\lambda_1\lambda_2 h + \lambda_2(u^3 - u)) = 0,\ h = \nabla^2 u
\]
\begin{itemize}
    \item Achieves phase separation modeling with 4.2e-3 MAE (vs. 6.7e-3 for tanh)
    \item Handles high-order derivatives through adaptive sin-exp blending (\(\alpha_{\text{sin}} = 0.82\pm0.03\))
\end{itemize}

\subsection*{Navier-Stokes Inverse Problems}
\begin{itemize}
    \item Simultaneous velocity-pressure prediction: RMSE 0.014 (vs. 0.021 for fixed activations)
    \item Viscosity estimation error <1.2\% from sparse sensor data
    \item Recovers pressure fields in cylinder wake flows with 92\% explained variance
\end{itemize}

\section*{13. Comparative Analysis of Adaptive Methods}
\begin{table}[h]
\centering
\caption{Performance Across Adaptive Activation Methods}
\begin{tabular}{l|c|c|c|c}
\textbf{Method} & \textbf{Burgers'} & \textbf{KdV} & \textbf{Cahn-Hilliard} & \textbf{SWEs} \\
\hline
SLAF [38] & 0.148 & 0.067 & 0.083 & 0.112 \\
PAU [39] & 0.095 & 0.042 & 0.061 & 0.089 \\
ACON [40] & 0.081 & 0.038 & 0.055 & 0.076 \\
ABU-PINN (Ours) & \textbf{0.075} & \textbf{0.029} & \textbf{0.042} & \textbf{0.063} \\
\end{tabular}
\end{table}

Key limitations:
\begin{itemize}
    \item SLAF suffers gradient explosion in high-order systems (Cahn-Hilliard)
    \item PAU exhibits discontinuities in second derivatives (invalid for SWE residuals)
    \item ACON lacks diversity for periodic systems (KdV phase error 12\% vs. 3.8\%)
\end{itemize}

\section*{14. Neural Tangent Kernel Analysis}
The enhanced performance of ABU-PINN is reflected in NTK eigenanalysis:
\begin{itemize}
    \item \textbf{Condition number improvement}: 1.2e3 vs. 3.4e3 for fixed tanh
    \item Dominant eigenvalues align with PDE frequency spectra (Fig. 17 in \cite{abu_pinn})
    \item Adaptive blending reduces kernel mismatch between PDE terms by 62\%
\end{itemize}

\section*{15. Conclusion}
The integration of adaptive activation functions and conservative formulations has significantly advanced PINNs' capabilities in hydrodynamics:
\begin{itemize}
    \item REAct achieves \textbf{3-order MSE reduction} on heat transfer vs. tanh
    \item ABU-PINN enables multi-physics modeling with <1\% parameter estimation error
    \item Conservative PINNs match FEM accuracy (8.9e-3 \(L_2\)) while maintaining GPU acceleration
\end{itemize}

Persisting challenges:
\begin{itemize}
    \item Shock handling (\(\nu < 0.001/\pi\)) requires hybrid WENO-PINN approaches
    \item Long-term error accumulation (12\% over 20 time units) in free-surface flows
    \item High-order systems (Cahn-Hilliard) demand deeper architecture integration
\end{itemize}

Future directions should focus on:
\begin{itemize}
    \item Coupling adaptive activations with uncertainty quantification frameworks
    \item Developing discontinuous Galerkin-inspired PINNs for shocks
    \item Operator learning for multi-scale bathymetry effects
\end{itemize}
