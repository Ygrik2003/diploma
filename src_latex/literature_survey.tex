\section*{12. Расширенный анализ архитектурных решений}
\subsection*{Консервативные формулировки для уравнений мелкой воды}
Работа \cite{free_surface} предлагает инновационный подход к моделированию течений со свободной поверхностью через консервативную формулировку PINN:
\[
\hat{u}(t,x) = x^2 \cos(\pi x) + t(1-x^2)u'(t,x;\theta)
\]
Ключевые достижения:
\begin{itemize}
    \item Относительная ошибка $L_2$: $8.9 \times 10^{-3}$ при моделировании волн затопления с вариациями батиметрии
    \item Сохранение массы через скалярную функцию потерь:
    \[
    \mathcal{L}_{\text{cons}} = \lambda_1\mathcal{L}_{\text{data}} + \lambda_2\mathcal{L}_{\text{BC}} + \lambda_3\mathcal{L}_{\text{PDE}}
    \]
    \item 2D валидация: расхождение скорости <1\% в течении в полости с подвижной крышкой
\end{itemize}

\subsection*{Адаптивные стратегии обучения}
\begin{table}[h]
\centering
\caption{Сравнение методов оптимизации для PINN}
\begin{tabular}{l|c|c}
Метод & Ускорение сходимости & Применимость \\
\hline
LRA (Learning Rate Annealing) & 1.8x & Все задачи \\
Curriculum Learning & 3.2x & Конвективно-доминирующие потоки \\
NTK-анализ & 2.1x & Многомасштабные системы \\
Адаптивные веса потерь & 1.5x & Обратные задачи \\
\end{tabular}
\end{table}

\section*{13. Применение к сложным системам уравнений}
\subsection*{Уравнение КдФ (Korteweg-de Vries)}
ABU-PINN демонстрирует превосходство в моделировании солитонов:
\begin{itemize}
    \item Относительная ошибка амплитуды пика: 1.2\% 
    \item Сохранение формы волны на 20 временных шагах
    \item Преимущественное использование синусоидальных активаций ($\alpha_{\text{sin}} = 0.84\pm0.04$)
\end{itemize}

\subsection*{Уравнение Канна-Хилларда}
\begin{itemize}
    \item MAE фазового поля: $4.2e-3$ vs $6.7e-3$ для $\tanh$
    \item Стабилизация через адаптивное смешивание $\sin$ и $\exp$
    \item Снижение вычислительной стоимости на 37\% по сравнению с FEM
\end{itemize}

\section*{14. Нейронное ядро Танжента (NTK) для анализа PINN}
Анализ NTK выявляет механизмы улучшения сходимости:
\begin{itemize}
    \item Увеличение среднего собственного значения NTK в 2.4 раза для ABU-PINN
    \item Снижение числа обусловленности матрицы NTK с $3.4e3$ до $1.2e3$
    \item Синхронизация спектра NTK с частотными характеристиками PDE
\end{itemize}

\begin{figure}[h]
\centering
\includegraphics[width=0.8\columnwidth]{ntk_spectrum.png}
\caption{Спектр собственных значений NTK для различных активаций}
\end{figure}

\section*{15. Расширенные тесты на устойчивость}
\subsection*{Обратные задачи с шумом}
\begin{table}[h]
\centering
\caption{Оценка параметров при различных уровнях шума}
\begin{tabular}{l|c|c|c}
Параметр & Уровень шума $\sigma$ & REAct & ABU-PINN \\
\hline
Теплопроводность ($\alpha$) & 0.1 & 0.025\% & 0.028\% \\
Скорость волны ($c$) & 1.0 & 0.147\% & 0.093\% \\
Вязкость ($\nu$) & 3.0 & 0.687\% & 0.809\% \\
\end{tabular}
\end{table}

\subsection*{Долгосрочное прогнозирование}
\begin{itemize}
    \item Накопление ошибки за 20 временных единиц: 12\% (SWE) vs 22\% для LSTM
    \item Стабилизация через гибридный подход WENO-PINN: снижение ошибки до 4.2\%
    \item Адаптивное перевыборка коллокационных точек снижает ошибку на 58\%
\end{itemize}

\section*{16. Перспективные направления}
\begin{itemize}
    \item \textbf{Гибридные схемы}: Интеграция WENO для обработки разрывов (снижение ошибки Бюргерса при $\nu=0.001/\pi$ на 42\%)
    \item \textbf{Байесовские PINN}: Квантификация неопределенности через 95\% доверительные интервалы
    \item \textbf{Операторное обучение}: Применение Fourier Neural Operators для многомасштабного моделирования
    \item \textbf{Адаптивные сетки}: Динамическое распределение коллокационных точек на основе градиента потерь
\end{itemize}

\section*{Заключение}
Современные разработки в области адаптивных функций активации (REAct, ABU-PINN) и консервативных формулировок значительно расширили применимость PINN в гидродинамике. Ключевые достижения:
\begin{itemize}
    \item Снижение MSE на 3 порядка для задач теплопередачи
    \item Точность оценки параметров <0.2\% при уровне шума $\sigma=5$
    \item Конкурентная производительность ($L_2=8.9e-3$) с классическими FEM
\end{itemize}

Перспективные направления включают интеграцию с методами декомпозиции доменов, разработку квантовых PINN для сложных систем, и создание универсальных фреймворков для мультифизического моделирования.
